\documentclass[12pt,a4paper,fleqn]{article}
\usepackage[utf8]{inputenc}
\usepackage{amssymb, amsmath, multicol}
\usepackage[russian]{babel}
\usepackage{concmath}
\usepackage{euler}
\usepackage{tikz}
\usepackage{paracol}
\usetikzlibrary{arrows}

\oddsidemargin=-15.4mm
\textwidth=190mm
\headheight=-32.4mm
\textheight=277mm
\tolerance=100
\parindent=0pt
\parskip=8pt
\pagestyle{empty}

\newtheorem{exersize}{Задача}

\begin{document}
\begin{center}
	\bf \Large ДОМАШНЕЕ ЗАДАНИЕ
	
	\bf \Large ТЕОРИЯ ГРАФОВ (PHYSTECH.INTERNATIONAL)
\end{center}

\begin{exersize}
	Существует ли граф с $8$-ю вершинами с степенями соответственно равными $2, 3, 3, 4, 4, 5, 5, 5$?
\end{exersize}

\begin{exersize}
	Докажите, что для любых смежных вершин $a$ и $b$ ребро, соединяющее их, будет принадлежать минимально к $deg (a) + deg (b) - |V|$ треугольникам в графе $G(V, E)$.
\end{exersize}

\begin{exersize}
	Докажите, что существует граф $G, |V| = 2k$, в котором нет троек вершин с одинаковой степенью. И при этом степень любой вершины не больше $k$ и не равна нулю.
\end{exersize}

\begin{exersize}
	Сколько рёбер в дополнении дерева на $n$ вершинах?	
\end{exersize}	

\begin{exersize}
	a) Ребра дерева окрашены в два цвета. Если в какой-то вершине сходятся ребра одного цвета, то можно их все перекрасить в другой цвет. Можно ли все дерево сделать одноцветным?
	
	b) Будем красить в два цвета не ребра, а вершины графа. Можно ли любое дерево раскрасить так, что любое ребро будет соединять вершины разных цветов?
	
	c) Докажите, что вершины графа можно раскрасить в два цвета тогда и только тогда, когда граф не содержит циклов нечетной длины.
\end{exersize}	

\begin{exersize}
	В стране 45 городов, некоторые из них соединены авиалиниями, принадлежащими трём авиакомпаниям. Известно, что даже если любая из авиакомпаний прекратит полёты, можно будет добраться из любого города в любой другой (возможно, с пересадками), пользуясь рейсами оставшихся двух компаний. Какое наименьшее число авиалиний может быть в стране?
\end{exersize}	

\begin{exersize}
	Нарисуйте три неизоморфных графа со степенной последовательностью (3, 3, 2, 2, 1, 1).
\end{exersize}

\begin{exersize}
	В группе из нескольких человек некоторые люди знакомы друг с другом, а некоторые нет. Каждый вечер один из них устраивает ужин для всех своих знакомых, на котором знакомит их друг с другом. После того, как каждый человек устроил хотя бы по одному ужину, оказалось, что какие-то два человека все еще не знакомы. Докажите, что они не познакомятся и на следующем ужине.
\end{exersize}

\begin{exersize}
	Верно ли, что при любом натуральном $k$, если в графе ровно $4k$ вершин имеют степень 5, а степени остальных~---~6, то нельзя удалить одно ребро так, чтобы этот граф распался на две изоморфные компоненты связности?
\end{exersize}


\begin{exersize}
	Один из простейших многоклеточных организмов~---~водоросль <<вольвокс>>~---~представляет собой сферическую оболочку,, сложенную семиугольными, шестиугольными и пятиугольными клетками (в каждой <<вершине>> сходятся три клетки). Биологи заметили, что пятиугольных клеток всегда ровно на 12 больше, чем семиугольных (всего клеток может быть несколько сотен и даже тысяч). Не можете ли вы объяснить этот странный факт?
\end{exersize}

\begin{exersize}
	Пусть все грани выпуклого многогранника~---~правильные $n$-угольники, и в каждой его вершине сходится ровно $k$ граней. Докажите, что тогда $\frac{1}{n} + \frac{1}{k} = \frac{1}{2} + \frac{1}{r}$, где $r$~---~число его рёбер.
\end{exersize}

\begin{exersize}
	При каких $n$ граф $K_n$ будет эйлеровым?
\end{exersize}

\begin{exersize}
	Верно ли, что можно на доске нарисовать произвольное количество касающихся окружностей, не отрывая мела? (если да, то докажите, что при произвольном расположении получится; если нет, то приведите контрпример)
\end{exersize}

\begin{exersize}
	В некоторой стране есть 2018 городов, соединенных подземными тоннелям, причём из любого города можно добраться до любого другого по подземным тоннелям, пройдя при этом все остальные города. Какое наименьшее число подземных тоннелей может быть в этой стране?
\end{exersize}

\begin{exersize}
	Сколько различных ориентированных графов можно получить из одного и того же простого графа $G(V, E)$ ориентацией его рёбер?
\end{exersize}

\begin{exersize}
	Докажите, что для любого натурального нечётного $n = 2k + 1$ существует турнир, в котором для всех вершин верно равенство
	$$outdeg (x) = k = indeg (x).$$
\end{exersize}

\begin{exersize}
	Докажите, что для любого турнира сумма квадратов входящих степеней всех вершин равна сумме квадратов выходящих степеней всех вершин.
\end{exersize}

\begin{exersize}
	В классе $15$ мальчиков. Известно, что каждый из них знаком с ровно тремя девочками, а каждая девочка знакома ровно с пятью мальчиками. Сколько в классе детей?
\end{exersize}

\begin{exersize}
	Докажите, что любое дерево является двудольным графом.
\end{exersize}

\begin{exersize}
	Докажите, что граф $Q_k$ является $k$-регулярным двудольным графом. Подсчитайте количество вершин и рёбер в таком графе.
\end{exersize}

\end{document}