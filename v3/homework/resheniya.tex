\documentclass[12pt,a4paper,fleqn]{article}
\usepackage[utf8]{inputenc}
\usepackage{amssymb, amsmath, multicol}
\usepackage[russian]{babel}
\usepackage{concmath}
\usepackage{euler}
\usepackage{tikz}
\usepackage{paracol}
\usetikzlibrary{arrows}
\usepackage{amsthm}

\oddsidemargin=-15.4mm
\textwidth=190mm
\headheight=-32.4mm
\textheight=277mm
\tolerance=100
\parindent=0pt
\parskip=8pt
\pagestyle{empty}

\newtheorem{exersize}{Задача}

\begin{document}
\begin{center}
	\bf \Large ДОМАШНЕЕ ЗАДАНИЕ
	
	\bf \Large ТЕОРИЯ ГРАФОВ (PHYSTECH.INTERNATIONAL)
\end{center}

\begin{exersize}
	Существует ли граф с $8$-ю вершинами с степенями соответственно равными $2, 3, 3, 4, 4, 5, 5, 5$?
	
	\emph{Решение.} По следствию из леммы о рукопожатиях - не может, так как в таком графе будет нечетное число нечетных вершин.
\end{exersize}

\begin{exersize}
	Докажите, что для любых смежных вершин $a$ и $b$ ребро, соединяющее их, будет принадлежать минимально к $deg (a) + deg (b) - |V|$ треугольникам в графе $G(V, E)$.
	
\begin{proof}
	Пусть множество $X$~---~множество вершин, смежных с $a$. Аналогично множество $Y$~---~с $b$. Тогда $$|X| = deg(a) - 1, |Y| = deg(b) - 1, |X \cup Y| \leqslant |V| - 2.$$
	Следовательно, $|X \cap Y| \geqslant deg (a) + deg (b) - |V|.$
\end{proof}
\end{exersize}

\begin{exersize}
	Докажите, что существует граф $G, |V| = 2k$, в котором нет троек вершин с одинаковой степенью. И при этом степень любой вершины не больше $k$ и не равна нулю.

\begin{proof}
	Заметим, что в таком графе должно быть ровно по две вершины каждой степени. Проведем доказательство по индукции. База очевидна. В шаге инукции будет новые две вершины соединять между собой, а одну из них еще с одной вершиной из каждой пары.
\end{proof}
\end{exersize}

\begin{exersize}
	Сколько рёбер в дополнении дерева на $n$ вершинах?	
	
	\emph{Решение.} Всего ребер в полном графе $\frac{n(n-1)}{2}$, в дереве ребер $n-1$. Вычитаем и получаем число $\frac{(n-1)(n-2)}{2}$.
\end{exersize}	

\begin{exersize}
	a) Ребра дерева окрашены в два цвета. Если в какой-то вершине сходятся ребра одного цвета, то можно их все перекрасить в другой цвет. Можно ли все дерево сделать одноцветным?
	
	b) Будем красить в два цвета не ребра, а вершины графа. Можно ли любое дерево раскрасить так, что любое ребро будет соединять вершины разных цветов?
	
	c) Докажите, что вершины графа можно раскрасить в два цвета тогда и только тогда, когда граф не содержит циклов нечетной длины.
	
	\emph{Решение.} a) да, делаем корневое дерево и фигачим индукцию по ярусам.
	
	b) да, любое дерево~---~двудольный граф
	
	c) в обе стороны очевидно
\end{exersize}	

\begin{exersize}
	В стране 45 городов, некоторые из них соединены авиалиниями, принадлежащими трём авиакомпаниям. Известно, что даже если любая из авиакомпаний прекратит полёты, можно будет добраться из любого города в любой другой (возможно, с пересадками), пользуясь рейсами оставшихся двух компаний. Какое наименьшее число авиалиний может быть в стране?
	
	\emph{Решение.} Связный граф с минимальным количеством ребер~---~дерево. Обозначим число авиалиний каждой компании соответственно $a, b, c$. Тогда из условия следует три неравенства $$a + b \geqslant 44, b + c \geqslant 44, c + a \geqslant 44.$$
	Суммируем три неравенства и получаем, что $a + b + c \geqslant 66$.
	
	(пупс обязательно должен привести пример графа, пожходящего под условие)
\end{exersize}	

\begin{exersize}
	Нарисуйте три неизоморфных графа со степенной последовательностью (3, 3, 2, 2, 1, 1).
	
	\emph{Решение.} В условии ошибка, там должна быть последовательность $(3, 2, 2, 1, 1, 1)$.
	
\begin{center}\begin{tikzpicture}
	\tikzstyle{every node}=[circle, draw, fill=red, inner sep=0pt, minimum width=6pt]	
	
	\draw (0, 0) node {} -- (1, 1) node {} -- (2, 1) node {};
	\draw (0, 0) node {} -- (1, 0) node {};
	\draw (0, 0) node {} -- (1, -1) node {} -- (2, -1) node {};	
\end{tikzpicture}
\;\ \;\ \;\ \;\ \;\ \;\ \;\ \;\ \;\ \;\ 
\begin{tikzpicture}
	\tikzstyle{every node}=[circle, draw, fill=red, inner sep=0pt, minimum width=6pt]	
	
	\draw (0, 0) node {} -- (1, 1) node {} -- (-1, 1) node {} -- (0, 0) node {} -- (0, -1) node {};
	\draw (1, -1) node {} -- (2, 0) node {};	
\end{tikzpicture}
\;\ \;\ \;\ \;\ \;\ \;\ \;\ \;\ \;\ \;\ 
\begin{tikzpicture}
	\tikzstyle{every node}=[circle, draw, fill=red, inner sep=0pt, minimum width=6pt]	
	
	\draw (0, 0) node {} -- (1, 0) node {} -- (2, 0) node {} -- (3, 0) node {};
	\draw (0, 0) node {} -- (1, 1) node {};
	\draw (0, 0) node {} -- (1, -1) node {};	
\end{tikzpicture}\end{center}
\end{exersize}

\begin{exersize}
	В группе из нескольких человек некоторые люди знакомы друг с другом, а некоторые нет. Каждый вечер один из них устраивает ужин для всех своих знакомых, на котором знакомит их друг с другом. После того, как каждый человек устроил хотя бы по одному ужину, оказалось, что какие-то два человека все еще не знакомы. Докажите, что они не познакомятся и на следующем ужине.
	
\begin{proof}
	Нужно рассмотреть две связанные вершины. Тогда с каждым вечером знакомств путь будет укорачиваться. В конце концов все связанные вершины станут смежными. Следовательно, после всех вечеров только не связанные вершины будут не смежны. Очевидно, что несвязанные вершины не смогут стать смежными.
\end{proof}
\end{exersize}

\begin{exersize}
	Верно ли, что при любом натуральном $k$, если в графе ровно $4k$ вершин имеют степень 5, а степени остальных~---~6, то нельзя удалить одно ребро так, чтобы этот граф распался на две изоморфные компоненты связности?
	
	\emph{Решение.} Заметим, что какое бы ребро мы не удалили бы, то в новом графе будет $4m+2$ вешины степени $5$. Следовательно, так как степенные последовательности у изоморфных компонент совпадаают, то в каждой должно быть ровно $2m+1$ вершина нечетной степени. Противоречие (не удовлетворяет лемме о рукопожатиях).
\end{exersize}


\begin{exersize}
	Один из простейших многоклеточных организмов~---~водоросль <<вольвокс>>~---~представляет собой сферическую оболочку,, сложенную семиугольными, шестиугольными и пятиугольными клетками (в каждой <<вершине>> сходятся три клетки). Биологи заметили, что пятиугольных клеток всегда ровно на 12 больше, чем семиугольных (всего клеток может быть несколько сотен и даже тысяч). Не можете ли вы объяснить этот странный факт?
	
	\emph{Решение.} Допустим, что семиугольных, шестиугольных и пятиугольных клеток в нас $n_7, n_6, n_5$ соответствтенно. Тогда так как любой многогранник планирен, то 
	$$|F| = n_7 + n_6 + n_5, |E| = \frac{7n_7 + 6n_6 + 5n_5}{2}, |V| = \frac{7n_7 + 6n_6 + 5n_5}{3}.$$
	Подставляем в формулу Эйлера
	$$(n_7 + n_6 + n_5) - \frac{7n_7 + 6n_6 + 5n_5}{6} = 2 \Rightarrow n_5 = n_7 + 12.$$
\end{exersize}

\begin{exersize}
	Пусть все грани выпуклого многогранника~---~правильные $n$-угольники, и в каждой его вершине сходится ровно $k$ граней. Докажите, что тогда $\frac{1}{n} + \frac{1}{k} = \frac{1}{2} + \frac{1}{r}$, где $r$~---~число его рёбер.
	
\begin{proof}
	Пусть в многограннике $|F| = m$ граней. Тогда число вершин $|V| = \frac{nm}{k}$, а число ребер $|E| = \frac{nm}{2} = r$. Тогда $m = \frac{2r}{n}$, $|V| = \frac{2r}{k}$.
	
	Следовательно, формула Эйлера имеет следующий вид
	$$\frac{2r}{n} + \frac{2r}{k} - r = 2.$$
	Делим на $2r$ и переносим одно слагаемое вправо.
\end{proof}
\end{exersize}

\begin{exersize}
	При каких $n$ граф $K_n$ будет эйлеровым?
	
	\emph{Решение.} Все вершины должны быть четными. Следовательно, при нечетных $n > 2$.
\end{exersize}

\begin{exersize}
	Верно ли, что можно на доске нарисовать произвольное количество касающихся окружностей, не отрывая мела? (если да, то докажите, что при произвольном расположении получится; если нет, то приведите контрпример)
	
	\emph{Решение.} Да, можно. Фактически это будет одна вершина с произвольным числом петель.
\end{exersize}

\begin{exersize}
	В некоторой стране есть 2018 городов, соединенных подземными тоннелям, причём из любого города можно добраться до любого другого по подземным тоннелям, пройдя при этом все остальные города. Какое наименьшее число подземных тоннелей может быть в этой стране?
	
	(если вам попробуют сдать эту задачу~---~перекреститесь и отправьте человека ко мне~---~Диме Гущину)
\end{exersize}

\begin{exersize}
	Сколько различных ориентированных графов можно получить из одного и того же простого графа $G(V, E)$ ориентацией его рёбер?
	
	\emph{Решение.} У каждого ребра могут быть два направления, следовательно $2^{|E|}$.
\end{exersize}

\begin{exersize}
	Докажите, что для любого натурального нечётного $n = 2k + 1$ существует турнир, в котором для всех вершин верно равенство
	$$outdeg (x) = k = indeg (x).$$
	
	\emph{Решение.} Расположим вершины на окружности и пустим ребра из каждой вершины в следующие за ней по часовой стрелке $k$ вершин.
\end{exersize}

\begin{exersize}
	Докажите, что для любого турнира сумма квадратов входящих степеней всех вершин равна сумме квадратов выходящих степеней всех вершин.
	
\begin{proof}
	Для любой вершины $a$ турнира $n-1 = outdeg(a) + indeg(a)$. Кроме того, по лемме о взятках $$\sum_{a \in V} outdeg (a) = \sum_{a \in V} indeg (a).$$
	Возводим в квадрат и подставляем равенство выше, получаем, что нам надо доказать, что 
	$$\sum_{a \in V, b \in V,  a \neq b} (n-1 - indeg(a))(n-1 - indeg(b)) = \sum_{a \in V, b \in V,  a \neq b} indeg(a)indeg(b).$$
	
	Преобразуем $$\frac{n(n-1)^3}{2} - \sum_{a \in V} (n-1)^2 indeg(a) = 0.$$
	Последнее равенство выполняется в силу леммы о взятках.
\end{proof}
\end{exersize}

\begin{exersize}
	В классе $15$ мальчиков. Известно, что каждый из них знаком с ровно тремя девочками, а каждая девочка знакома ровно с пятью мальчиками. Сколько в классе детей?
		
	\emph{Решение.} Рассмотрим двудольный граф. Если мальчиков $x$, а девочек $y$, то $3x = 5y$. Следовательно, в классе $9$ девочек, а детей $24$.
\end{exersize}

\begin{exersize}
	Докажите, что любое дерево является двудольным графом.
	
\begin{proof}
	Сделаем его корневым. Нечетные ярусы в одну долю, четные~---~в другую.
\end{proof}
\end{exersize}

\begin{exersize}
	Докажите, что граф $Q_k$ является $k$-регулярным двудольным графом. Подсчитайте количество вершин и рёбер в таком графе.
	
\begin{proof}
	Все вершины $Q_k$ куба можно занумеровать битовыми строками длины $k$. А соединены будут те, которые будут отличаться в одном бите. Следовательно, каждая вершина будет иметь $k$ ребер, а значит, граф будет регулярным.
	
	Из абзаца выше следует, что вершин $2^k$, а ребер $k2^{k-1}$.
\end{proof}
\end{exersize}

\end{document}