\sloppy
\chapter*{От автора}
\addcontentsline{toc}{chapter}{От автора}

\epigraph{\itshape Они слили вместе религию, искусство и науку: 
ведь наука в конечном счете --- исследование чуда, коего мы не в силах объяснить,
а искусство~---~толкование этого чуда.}{--- Рэй Брэдбери, \textit{Марсианские хроники}}

	Во время длительного процесса редактирования книги я преследовал постоянно цель 
	уберечь читателя от траты времени на те факты, с которыми он уже знаком, добавляя для этого много сопутствующих примечаний. 
	Из-за этого некоторые части пособия раздувались так, что хотелось взять метлу с совком и выгрести весь мусор из них. 
	
	Поэтому я решил еще до начала пособия предупредить читателя: если во время чтения пособия 
	вы видите знакомый вам материал, не стоит тратить на него много времени, а, вероятно, стоит даже пропустить его. 
	Этот же совет касается тех разделов, которые будут вызывать у вас мигрени, желание~---~пойти отдохнуть. 
	Следуя этому совету, вы сможете, как на ковре-самолете, пролететь над теорией графов и алгоритмов, 
	<<насладится видами>> и не влететь на скорости в непреодолимые для вас <<стены>>.
	
	Первая глава наполнена материалом, необходимым для новичков и всем тем, 
	кому нравится не только изучать математику, но и обозревать истории, возникающие в процессе становления этой науки.
	
	Вторая глава была создана в помощь мне для проведения курса теории графов в <<Phystech.International $2018$>> 
	и в течение этого лагеря эволюционировала, приобретая много нового, что я почерпнул из общения со школьниками.
	Вот список тех, кто первым прослушал этот курс: Саморуков М. А., Прокопец П. А., Верещагина С. В., Шиндян Д. С.,
	Троян-Головян В. Д., Михеева Т. А., Кушнир А. В., Волгина Е. К.
	
	Третья глава пока что находится на очень ранней стадии разработки и предварительно она будет содержать материал, 
	собранный мной в процессе прослушивания курса алгоритмов на $1$ курсе ФУПМа МФТИ.
	
	\emph{Sapere Aude!} (Дерзайте знать!)

\begin{flushright}
\textit{Дима Г.}

\textit{\today}
\end{flushright}