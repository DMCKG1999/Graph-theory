\section{Экстремальные задачи}

\begin{exersize}[Агаханов Н.Х., Богданов И.И., Кожевников П.А., Подлипский О.К., Терешин Д.А. <<Всероссийские олимпиады школьников по математике 1993~---~2009: Заключительные этапы>>]
	В стране несколько городов, некоторые пары городов соединены дорогами. При этом из каждого города выходит хотя бы $3$ дороги. Докажите, что существует циклический маршрут, длина которого не делится на $3$.
\end{exersize}	 

\begin{exersize}[Агаханов Н.Х., Богданов И.И., Кожевников П.А., Подлипский О.К., Терешин Д.А. <<Всероссийские олимпиады школьников по математике 1993~---~2009: Заключительные этапы>>]
	В стране $2001$ город, некоторые пары городов соединены дорогами, причем из каждого города выходит хотя бы одна дорога и нет города, соединенного дорогами со всеми остальными. Назовем множество городов $D$ \emph{доминирующим}, если любой не входящий в $D$ город соединен дорогой с одним из городов множества $D$. Известно, что в любом доминирующем множестве хотя бы $k$ городов. Докажите, что страну можно разбить на $2001-k$ республик так, что никакие два города из одной республики не будут соединены дорогой.
\end{exersize}	 

\begin{exersize}[Агаханов Н.Х., Богданов И.И., Кожевников П.А., Подлипский О.К., Терешин Д.А. <<Всероссийские олимпиады школьников по математике 1993~---~2009: Заключительные этапы>>]
	В стране $N$ городов. Некоторые пары из них соединены беспосадочными двусторонними авиалиниями. Оказалось, что для любого $k$ ($2 \leqslant k \leqslant N$) при любом выборе городов количество авиалиний между этими городами не будет превосходить $2k-2$. Докажите, что все авиалинии можно распределить между двумя авиакомпаниями так, что не будет замкнутого авиамаршрута, в котором все авиалинии принадлежат одной компании.
\end{exersize}	 