\section{Графы как метрическое пространство}


\begin{definition}
	\emph{Расстоянием $d(a, b)$ между вершинами $a$ и $b$} называется длина наименьшего пути, соединяющего их.
\end{definition}
	
	Заметим, что такой путь всегда простой. Кроме того, чтобы это понятие распространялось на несвязные графы, принято считать, что для несвязанных вершин $c$ и $e$ расстояние $d(c, e) = \infty$.
	

\begin{exersize}[Агаханов Н.Х., Богданов И.И., Кожевников П.А., Подлипский О.К., Терешин Д.А. <<Всероссийские олимпиады школьников по математике 1993~---~2009: Заключительные этапы>>]
	В стране несколько городов, некоторые пары городов соединены дорогами, причем между любыми двумя городами существует единственный несамопересекающийся путь по дорогам. Известно, что в стране ровно $100$, из которых выходит по одной дороге. Докажите, что можно построить $50$ новых дорог так, тчо после этого даже при закрытии любой дороги можно будет из любого города попасть в любой другой.
\end{exersize}	 

\begin{exersize}[Агаханов Н.Х., Богданов И.И., Кожевников П.А., Подлипский О.К., Терешин Д.А. <<Всероссийские олимпиады школьников по математике 1993~---~2009: Заключительные этапы>>]
	В стране $1993$ города, и из каждого выходит не менее $93$ дорог. Известно, что из любого города можно проехать по дорогам в любой другой. Докажите, что это можно сделать не более, чем с $62$ пересадкамию (Дорога соединяет между собой два города.)
\end{exersize}	 

\begin{exersize}[Агаханов Н.Х., Богданов И.И., Кожевников П.А., Подлипский О.К., Терешин Д.А. <<Всероссийские олимпиады школьников по математике 1993~---~2009: Заключительные этапы>>]
	В стране $100$ городов, некоторые пары городов соединены дорогами. Для любых четырех городов существуют хотя бы две дороги между ними. Известно, что не существует маршрута, проходящего по каждому городу ровно один раз. Докажите, что можно выбрать два города таким образом, чтобы любой из оставшихся городов был соединен дорогой хотя бы с одним из двух выбранных городов.
\end{exersize}	

\begin{exersize}[Агаханов Н.Х., Богданов И.И., Кожевников П.А., Подлипский О.К., Терешин Д.А. <<Всероссийские олимпиады школьников по математике 1993~---~2009: Заключительные этапы>>]
	В стране некоторые пары городов соединены дорогами, которые не пересекаются вне городов. В каждом городе установлена табличка, на которой указана минимальная длина маршрута, выходящего из этого города и проходящего по всем остальным городам страны (маршрут может проходить по некоторым городам больше одного раза и не обязан возвращаться в исходный город). Докажите, что любые два числа на табличках отличаются не более, чем в полтора раза.
\end{exersize}	  

\begin{exersize}
	Удаленностью вершины дерева назовём сумму расстояний от неё до всех остальных вершин. Докажите, что в дереве, у которого есть две вершины с удаленностями, отличающимичся на 1,~---~нечётное число вершин.
\end{exersize}	

\begin{exersize}
	Докажите, что в любом графе $G$ для расстояния $d(a, b)$ выполняется неравенство треугольника, то есть 
	$$d(a, b) + d(b, c) \leqslant d(a,c) \;\ \forall \!\ a, b, c \in V.$$
\end{exersize}	

\begin{exersize}
	Пусть в графе $G$ $45$ вершин и степень каждой из них не меньше $22$. Докажите, что любые две вершины $a$ и $b$ графа $G$ либо смежны, либо $d (a, b) = 2$. 
\end{exersize}	