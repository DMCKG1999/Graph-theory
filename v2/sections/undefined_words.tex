\section{О неопределяемых понятиях}

\mysubsection{Обозначения}
	В первую очередь, я решил указать списком все обозначения, которые мы дальше будем использовать в надежде на то, что читатель, наткнувшийся на непонятный для него знак в тексте, сможет обратится к этому списку и обновить свои знания.
\begin{itemize}
\item $\Rightarrow$~---~следствие, импликация; $A \Rightarrow B$ означает, что если верно $A$, то верно и $B$,
\item $\Leftrightarrow$~---~равносильность, эвивалентсность; $A \Leftrightarrow B$ означает, что $A$ верно тогда и только тогда, когда верно $B$, 
\item $\wedge$~---~конъюнкция, знак <<и>>; $A \wedge B$ истинно тогда и только тогда, когда и $A$, и $B$ истинны,
\item $\vee$~---~дизъюнкция, знак <<или>>; $A \vee B$ истинно, когда $A$ истинно или $B$ истинно,
\item $\neg$~---~отрицание; $\neg A$ истинно, если $A$ ложно,
\item $\forall$~---~квантор всеобщности, <<для каждого>>; $\forall \!\ a, P(a)$ означает, что для всякого $a$ верно $P(a)$,
\item $\exists$~---~квантор существования, <<существует>>; $\exists \!\ a, P(a)$ означает, что существует такое $a$, что $P(a)$ верно,
\item $\exists!$~---~<<существует единственый>>; $\exists! \!\ a, P(a)$ означает, что существует единственное такое $a$, что $P(a)$ верно,
\item $\colon$~---~такое, что; $a: P(a)$ означает, что мы рассматриваем только такие $a$, что $P(a)$ верно,
\item $\colon=$~---~по определению равно; $a \colon = b$ означает, что по определению $a$ равно $b$,
\item $\mapsto$~---~выполняется; $\forall \!\ a \mapsto P(a)$ означает, что для любого $a$ верно $P(a)$,
\item $\displaystyle\sum$~---~сумма; $\displaystyle\sum_{a \in A} P(a)$ равно сумме значений $P(a)$,
\item $\displaystyle\prod$~---~произведение; $\displaystyle\prod_{a \in A} P(a)$ равно произведению значений $P(a)$.
\end{itemize}

	В силу того, что некоторая часть читателей может быть не знакома с такими понятиями, как множество и отображения, я решил, что не будет лишним сделать небольшое отступление в начале и немного рассказать о них.
	
\mysubsection{Множества}

	Вспомним уроки геометрии в школе, где, возможно, не всем из нас, но многим приходилось учить аксиомы планиметрии или сталкиваться с ними. Кроме этих аксиом, также были и неопределяемые понятия: например, точка, плоскость. И это не исчерпывает список того, что закладывается в фундамент математики: нам также, не объясняя, говорили о принципе наложения фигур друг на друга.
	
	Есть отдельные области в математики, которые занимаются обсуждением того, что лежит в самых низах. К ним можно отнести и теорию множеств, и теория моделей, и теорию доказательств. За всеми этими словами стоят гигантские тексты, в которых изложено неизмеримое множество утверждений. Мы не будем вдаваться во всё это, а остановимся на том материале, который нам позволит грамотно воспринимать всю оставшуюся книгу.
	
	Суть множества~---~набор элементов, не имеющий никакой структуры. Сами элементы могут иметь произвольную природу. Будем говорить, что \emph{элемент $a$ принадлежит множеству $A$}, если $a$ присутствует в наборе элементов этого множества, и писать $a \in A$. Множество однозначно задается своими элементами, то есть $A = B$ тогда и только тогда, когда любой элемент $A$ принадлежит множеству $B$, и наоборот.
	
	Заметим, что в определении множества мы использовали двойное вложение, то есть $A \subset B$ и $B \subset A$. В общем случае множество $A$, вложенное в множество $B$, называется его \emph{подмножеством}.
	
	Отдельно уточним то, что каждый элемент может представлять из себя тоже отдельное множество. В этом смысле множества могут походить на функциональный язык LISP, после работы с которым в кошмарах программиста еще долго будут сниться круглые скобочки.
	
	Нам будут встречаться \emph{числовые множества}:
\begin{itemize}
	\item натуральные числа ($\BN$)~---~$1, 2, 3, 4, 5, 6, 7, 8, 9, \dots$
	\item целые числа ($\BZ$)~---~$\dots, -4, -3, -2, -1, 0, 1, 2, 3, 4, 5, 6, \dots$
	\item рациональные числа ($\BQ$)~---~$\dots,\frac{1}{2}, 2, 3, -\frac{3}{17}, -5, \frac{4}{53}, 0.(3), \dots$
	\item действительные числа ($\BR$)~---~$\dots,e, -\frac{1}{2}, 16, -\sqrt{3}3, -\frac{17}{3}, -5, \frac{4}{53}, 0.(3), \pi, \dots$
	\item комплексные числа ($\BC$)~---~$\dots, ie, 1 + i, 16i, -\sqrt{-3}3, -\frac{5}{24}, -7, \frac{97}{53 + i}, \frac{2+i}{3+i}, \pi + i, \dots$
\end{itemize}
	
	А некоторые множества мы будет сами перечислять, то есть, например, множество покупок в продуктовом может выглядеть следующим образом
	$$\lbrace \text{мясо}, \:\ \text{апельсины}, \:\ \text{гречка}, \:\ \text{молоко}, \:\ \text{овощи}, \:\ \text{рыба}, \:\ \text{хлеб}, \:\ \text{квас}, \dots \rbrace.$$
	
	А множество оценок, которые ставит преподаватель в течение года, будет выглядеть так
	$$\lbrace 5, 4, 3, 2, 1\rbrace.$$
	
	\emph{Пустое множество} не содержит ни одного элемента и обозначается $\varnothing$.
	
\begin{testquestion}
	Сколько элементов в множестве $\lbrace \lbrace 1, 3\rbrace, 4, \lbrace 2, 4, 5\rbraceб \varnothing \rbrace$?
\end{testquestion}	

	Кроме явных представлений множеств, их также можно определить посредством введения условий, которое обычно пишут в окружении фигурных скобок. Например, множество элементов, кратных числу $p$ обозначается так
	$$p\BZ = \lbrace a \colon a=pk, k \in \BZ\rbrace.$$

\begin{testquestion}
	Как называется это множество: $\lbrace a \colon a^2 + 1 = 0, a \in \BR\rbrace$?
\end{testquestion}	
	
	
	Заметим, что далее мы будем подразумевать, что имеем дело с обычными, а не \emph{мультимножествами}, в которых допускается повторение элементов. К тому же нельзя забывать о том, что главное отличие множества от кортежа в том, что оно не упорядочено, то есть 
	$$\lbrace 5, 4, 3, 2, 1\rbrace = \lbrace 3, 2, 4, 5, 1\rbrace = \lbrace 1, 2, 3, 4, 5\rbrace = \lbrace 4, 5, 3, 1, 2\rbrace.$$
	
\mysubsection{Операции на множествах}	
	
	Множества подобны существительным в языке. Именно так же, как и последние нуждаются в глаголах, множества нуждаются в операциях, которые и определяли бы всю дальнейшую работу с ними.
	
	\emph{Объединением множеств $A$ и $B$} называется множество, состоящее из элементов множества $A$ и множества $B$, и обозначается $A \cup B$. Другими словами,
	$$a \in A \cup B \Leftrightarrow a \in A \vee a \in B.$$
	
	\emph{Пересечением множеств $A$ и $B$} называется множество, состоящее из элементов, которые встречаются как в $A$, так и в $B$, и обозначается $A \cap B$. Иначе говоря,
	$$a \in A \cap B \Leftrightarrow a \in A \wedge a \in B.$$
	
	\emph{Разностью множеств $A$ и $B$} называется множество, состоящее из элементов, которые встречаются в $A$, но не встречаются в $B$, и обозначается $A \backslash B$:
	$$a \in A \backslash B \Leftrightarrow a \in A \wedge a \notin B.$$
	
	Также отдельно выделяют \emph{симметрическую разность множеств $A$ и $B$}, подразумевая под ней $(A \backslash B) \cup (B \backslash A)$ и обозначая $A \bigtriangleup B$.
	
\begin{testquestion}
	Что это за множество $(\BZ \bigtriangleup \BN) \cup \BN$?
\end{testquestion}

	Еще есть \emph{декартово произведение $A \times B$}~---~это множество упорядоченных пар $(a, b)$ таких, что $a \in A$, а $b \in B$. Часто используют обозначение $A^n$ для множества, полученного декартовым произведением $A$ множеств $n$ раз.
	
\mysubsection{Отображения}
	
	По сути \emph{отображение $f$} есть ни что иное, как правило, по которому мы каждому элементу множества $A$ ставим в соответствие единственный элемент множества $B$. В таком случае принято обозначение $f: A \to B$.
	
	Среди всевозможных отображений $f: A \to B$ выделяют следующие три их вида:

\begin{enumerate}
	\item \emph{инъекция}: $\forall \!\ a_1, a_2 \in A \colon a_1 \neq a_2 \mapsto f(a_1) \neq f(a_2)$,
	\item \emph{сюръекция}: $\forall \!\ b \in B \exists \!\ a \in A \colon f(a) = b$,
	\item \emph{биекция}: $\forall \!\ b \in B \exists! \!\ a \in A \colon f(a) = b$.
\end{enumerate}

\begin{testquestion}
	Каким будет отображение $f: \BN \to \BN, f(n) = 2n$?
\end{testquestion}

\begin{testquestion}
	Каким будет отображение $f: \BR \to \BR, f(x) = x \cdot \sin(x)$?
\end{testquestion}

	В речи и письме часто употребляют понятия: \emph{образ} $a$, подразумевая $f(a)$; \emph{прообраз} $b$, подразумевая $a \in A \colon f(a) = b$; \emph{образ отображения}~---~подмножество $B$, состоящее из всех образов элементов из $A$.