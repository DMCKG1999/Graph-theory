\newpage
\section{О раскраске графов}

\mysubsection{Задачи}

\begin{exersize}[Агаханов Н.Х., Богданов И.И., Кожевников П.А., Подлипский О.К., Терешин Д.А. <<Всероссийские олимпиады школьников по математике 1993~---~2009: Заключительные этапы>>]
	В стране $N$ городов. Между любыми двумя из них проложена либо автомобильная, либо железная дорога. Турист хочет объехать страну, побывав в каждом городе ровно один раз, и вернуться в город, с которого он начнет путешествие, и маршрут так, что ему придется поменять вид транспорта не более одного раза.
\end{exersize}	 

\begin{exersize}[Агаханов Н.Х., Богданов И.И., Кожевников П.А., Подлипский О.К., Терешин Д.А. <<Всероссийские олимпиады школьников по математике 1993~---~2009: Заключительные этапы>>]
	Улицы города Дужинска~---~простые ломаные, не пересекающиеся между собой во внутренних точках. Каждая улица соединяет два перекрестка и покрашена в один из трех цветов: белый, красный или синий. На каждом перекрестке сходятся ровно три улицы, по одной каждого цвета. Перекресток назыавется положительным, если при его обходе против часовой стрелки цвета улиц идут в следующем порядке: белый, синий, красный, и отрицательным в противном случае. Докажите, что разность между числом положительных и числом отрицательных перекрестков кратна четырем.
\end{exersize}	 

\begin{exersize}[Агаханов Н.Х., Богданов И.И., Кожевников П.А., Подлипский О.К., Терешин Д.А. <<Всероссийские олимпиады школьников по математике 1993~---~2009: Заключительные этапы>>]
	В стране $2000$ городов, некоторые пары городов соединены дорогами. Известно, что через любой город проходит не более $N$ различных несамопересекающихся циклических маршрутов нечетной длины. Докажите, что страну можно разделить на $N+2$ республики так, чтобы никакие два города из одной республики не были соединены дорогой.
\end{exersize}	 


\begin{exersize}[Агаханов Н.Х., Богданов И.И., Кожевников П.А., Подлипский О.К., Терешин Д.А. <<Всероссийские олимпиады школьников по математике 1993~---~2009: Заключительные этапы>>]
	В городе несколько площадей. Некоторые пары площадей соединены улицами с односторонним движением так, что с каждой площади можно выехать ровно по двум улицам. Докажите, что город можно разбить ан $1014$ районов так, чтобы улицами соединялись только площади из разных районов, и для любых двух районов все соединяющие их улицы были направлены одинаково (либо все из первого района во второй, либо наоборот).
\end{exersize}

\begin{exersize}[Агаханов Н.Х., Богданов И.И., Кожевников П.А., Подлипский О.К., Терешин Д.А. <<Всероссийские олимпиады школьников по математике 1993~---~2009: Заключительные этапы>>]
	В кабинете президента стоят $2004$ телефона, любые два из которых соединены проводом одного из четырех цветов. Известно, что провода всех четырех цветов присутствуют. Всегда ли можно выбрать несколько телефонов так, чтобы среди соединяющих их проводов встречались провода ровно трех цветов?
\end{exersize}	 

\begin{exersize}[Агаханов Н.Х., Богданов И.И., Кожевников П.А., Подлипский О.К., Терешин Д.А. <<Всероссийские олимпиады школьников по математике 1993~---~2009: Заключительные этапы>>]
	На бесконечном белом листе клетчатой бумаги конечное число клеток окрашено в черный цвет так, что у каждой черной клетки четное число ($0$, $2$ или $4$) белых клеток, соседних с ней по стороне. Докажите, что каждую белую клетку можно покрасить в красный или зеленый цвет так, чтобы у каждой черной клетки стало поровну красных и зеленых клеток, соседних с ней по стороне.
\end{exersize}	 	 

\begin{exersize}[Федоров Р. М., Канель-Белов А. Я., Ковальджи А. К., Ященко И. В. <<Московские математические олимпиады>>]
	Некоторый граф правильно раскрашен в $k$ цветов, причем его нельзя правильно раскрасить в меньшее число цветов. Докажите, что в этом графе существует путь, вдоль которого встречаются вершины всех $k$ цветов ровно по одному разу.
\end{exersize}	
% Бабичева Пособие 335 задача 7.3.31, 7.3.33