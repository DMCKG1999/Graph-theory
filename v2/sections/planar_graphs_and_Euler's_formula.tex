\newpage
\section{О планарных графах}

Ранее мы уже говорили о том, что вовсе не обязательно рассматривать графы только на плоскости. Можно представить себе и трёхмерный граф. Конечно, для каждого графа существует его плоская изоморфная реализация. Правда, рёбра у такого графа могут пересекаться.

\begin{definition}
	\emph{Плоским или планарным} называют граф, у которого существуют изоморфный граф, рёбра которого не пересекаются нигде, кроме вершин. Такой граф делит плоскость на области (включая внешнюю), называемые \emph{гранями}. Множество граней будем обозначать буквой $F$.
\end{definition}

\mysubsection{Формула Эйлера}

	В отличие от своих собратьев, планарный связный граф не может иметь произвольное количество ребер, вершин и граней. На них накладывается некоторое соотношение, которое называется \emph{формулой Эйлера}: $$|V| - |E| + |F| = 2.$$

\begin{proof}
	Воспользуемся методом математической индукции. В качестве счётчика возьмём количество граней.

	База индукции: плоский связный граф, у которого одна грань,~---~это дерево. Как было доказано выше, в дереве число ребер на один меньше числа вершин, таким образом 
$$|V| - |E| + |F| = |V| - (|V| - 1) + 1 = 2.$$
	База доказана.
	
	Теперь предполохим, что для всех плоских связных граней с $k$ гранями верна формула Эйлера, и докажем, что при $(k+1)$-й грани всё будет выполняться.
	
	Так как граней больше одной, то у нас граф содержит цикл. Давайте рассмотрим произвольное ребро произвольного цикла. Если его стереть, то количество граней уменьшится, но при этом граф останется связным. По предположению для нового графа будет выполнена формула Эйлера, то есть $|V| - (|E| - 1) + (|F| - 1) = 2$. Раскрывая скобки, получим искомое равенство.
\end{proof}
	
\mysubsection{$K_{3,3}$ и $K_5$}

	Наиболее знаменитый неплоские графы~---~это $K_5$ и $K_{3,3}$. Они оба непланарны. Кроме того, две теоремы~---~Куратовского, Вагнера~---~дают критерий планарности произвольного графа, сводя в некотором смысле задачу к поиску <<подграфа, изоморфного $K_5$ или $K_{3,3}$>>.
	
\begin{example}
	Обитатели трёх домов, в совместном владении которых находятся три колодца, пересcорились друг с другом и решили проложить к своей собственности непересекающиеся тропки~---~от каждого дома к каждому колодцу. Докажите, что это им не удастся...
	
	\emph{Доказательство.} Пусть колодца и дома будут вершинами графа, а тропки - ребрами. Тогда мы имеем дело с $K_{3, 3}$. Это непланарный граф. Следовательно, у них ничего не получится.
\end{example}

	На самом деле в этой задаче мы пользовались тем, что герои живут на планете или плоскости. Читатель может подумать на досуге над следующим вопросом: получилось ли у обитателей трёх домов проложить эти тропки, если бы они все жили на планете в форме бублика?
	
\mysubsection{Интересные факты}

	В некоторых задачах можно наткнуться на понятие выпуклой оболочки. Как таковое, оно не входит в теорию графов, однако полезно знать о нём и в некоторых задачах использовать.
	
	Допустим, что на плоскости отмечено произвольное (конечное) число точек. Тогда существует такой выпуклый многоугольник с вершинами в некоторых отмеченных точках, что все остальные точки лежат внутри него. Этот многоугольник и называется \emph{выпуклой оболочкой}.
	
	Планарные графы по определению можно изобразить на плоскости без самопересечения рёбер. Оказывается, что все планарные графы можно с тем же условием нарисовать и на сфере. Кроме того, обратное утверждение верно, то есть любой конечный граф на сфере можно изоморфно переместить в плоский граф.

$ $
\newline
	В итоге мы познакомились с планарными графами доказали формулу Эйлера и поговорили о полезных математических фактах, которые находятся вблизи от теории графов.
	
\mysubsection{Задачи}

\begin{exersize}
	В некоторой стране есть $n$ озер, которые соединены $k$ каналами. Из любого озера по каналам можно добраться в любое другое озеро. Сколько в этой стране островов?
\end{exersize}

\begin{exersize}
	Докажите, что для связного плоского графа выполняются неравенства:
	$$a)\!\ 2|E| \geqslant 3|F| \;\ \text{при} \;\ |V| \geqslant 2; \;\ b) \!\ |E| \leqslant 3|V| - 6  \;\ \text{при}  \;\ |V| \geqslant 3.$$
\end{exersize}

\begin{exersize}
	На плоскости отмечено несколько точек, никакие три из которых не лежат на одноц прямой. Двое по очереди соединяют отрезком две какие угодно еще не соединенные точки так, чтобы отрезки не пересекались нигде, кроме отмеченных точек. Проигрывает тот, кто не сможет сделать ход. Докажите, что один из играющих будет всегда выигрывать независимо от своей игры и игры соперника.
\end{exersize}

\begin{exersize}
	Один из простейших многоклеточных организмов~---~водоросль <<вольвокс>>~---~представляет собой сферическую оболочку,, сложенную семиугольными, шестиугольными и пятиугольными клетками (в каждой <<вершине>> сходятся три клетки). Биологи заметили, что пятиугольных клеток всегда ровно на 12 больше, чем семиугольных (всего клеток может быть несколько сотен и даже тысяч). Не можете ли вы объяснить этот странный факт?
\end{exersize}

\begin{exersize}
	Пусть все грани выпуклого многогранника~---~правильные $n$-угольники, и в каждой его вершине сходится ровно $k$ граней. Докажите, что тогда $\frac{1}{n} + \frac{1}{k} = \frac{1}{2} + \frac{1}{r}$, где $r$~---~число его рёбер.
\end{exersize}

\begin{exersize}(Канель-Белов А.Я., Ковальджи А.К. Как решают нестандартные задачи)
	Докажите, что в плоском графе найдется вершина, из которой выходит не более 5 ребер.
\end{exersize}

\begin{exersize}[Канель-Белов А.Я., Ковальджи А.К. Как решают нестандартные задачи]
	На плоскости дано $n$ точек, соединенных непересекающимися отрезками. Может ли оказаться так, что каждая точка будет соединена ровно с шестью другими?
	
\end{exersize}
