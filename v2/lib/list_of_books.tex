\sloppy
\chapter*{Список литературы}
\addcontentsline{toc}{chapter}{Список литературы}
\begin{enumerate}
\item Берж К. Теория графов и ее применения. --- М.: ИЛ, 1962.
\item Юшкевич А. П. История математики. С древнейших времен до начала нового времени. --- М.: Наука, 1970.
\item Кристофидес Н. Теория графов. Алгоритмический подход. --- М.: Мир, 1978.
\item Емеличев В. А., Мельников О.И., Сарванов В. И., Тышкевич Р. И. Лекции по теории графов --- М.: Наука, 1990.
\item Генкин С.А., Итенберг И.В., Фомин Д.В. Ленинградские математические кружки. --- К.: АСА, 1994.
\item Журавлёв Ю. И., Флёров Ю. А. Дискретный анализ. Ч. 1: учебное пособие. --- М.: МФТИ, 1999.
\item Спивак А.В. Математический праздник. --- М.: Бюро Квантум, 2004.
\item Выгодский М.Я. Справочник по элементарной математике. --- М.: АСТ:Астрель, 2006.
\item Канель-Белов А.Я., Ковальджи А.К. Как решают нестандартные задачи, 5-е издание. --- М.: МЦНМО, 2009.
\item Райгородский А.М. Экстремальные задачи теории графов и Интернет. --- Д.: Издательский Дом <<Интеллект>>, 2012.
\item Стюарт И. Великие математические задачи. --- М.: Альпина нон-фикшн, 2015.
\item Кормен Т., Лейзерсон Ч., Ривест Р., Штайн К. Алгоритмы: построение и анализ, 3-е издание. --- М.: Вильямс, 2016.
\item Виленкин Н. Я., Виленкин А. Н., Виленкин П. А. Комбинаторика, 6-е издание, стереотипное. --- М.: ФИМА; МЦНМО, 2017. 
\item Агаханов Н.Х., Богданов И.И., Кожевников П.А., Подлипский О.К., Терешин Д.А. Всероссийские олимпиады школьников по математике 1993~---~2009: Заключительные этапы, 4-е издание, стереотипное. --- М.: МЦНМНО, 2017. 
\item Федоров Р. М., Канель-Белов А. Я., Ковальджи А. К., Ященко И. В. Московские математические олимпиады, 3-е издание. --- М.: МЦНМНО, 2017. 
\item Бабичева Т.С., Бабичев С.Л., Жогов А. А., Яковлев И.В. Пособие по олимпиадной математике. Уровень А1. --- М.: Эдитус, 2018.
\item Омельченко А. В. Теория графов. -- М.: МЦНМО, 2018.
\item Харари Ф. Теория графов, 5-е издание. --- М.: ЛЕНАНД, 2018.
\end{enumerate}


