\newpage
\section{О сетях и потоках}

	Вспомним теорему Холла, доказанную в предыдущем параграфе. Можно заметить, что само доказательство, несмотря на его математическую строгость, не показывает нам, как можно для конкретного двудольного графа проверить: есть или нет совершенное паросочетание в нём? 
	
	Для этого нам придется перебрать все возможные непустые множества юношей, которых будет $2^n - 1$, если $n$~---~количество юношей. Как при рассмотрении задачи о коммивояжёре, этот способ необыкновенно затратный по ресурсам, поэтому не оптимальный. 
	
	Казалось бы минусы уже закончились, но нет. Кроме того, что таким способом мы потратим много времени, так ещё он неконструктивный, то есть не даст по итогу нам сами пары, в которых надо поженить юноша на девушке, а просто скажет, что так можно сделать.
	
	Поэтому чтобы уменьшить наши мучения, обсудим алгоритм, который сначала ещё раз докажет теорему о сватовстве, а потом ещё и предоставим нам короткий способ к поиску совершенного паросочетания.

	С чего начать? Конечно, с переформулирования задачи. Представим, что поиск паросочетания~---~это поиск функции на рёбрах графа, которая должна удовлетворять некоторым свойствам. Во-первых, её областью значений будет множество $\lbrace 0, 1\rbrace$, рёбрам, которые мы выбрали будет поставлена в соответствие единица, а остальным~---~ноль. Во-вторых, надо наложить на эту функцию условия: сумма по рёбрам для каждой вершины, изображающей юношу, должна равняться единице, а соответствующая сумма для вершин, изображающих девушек, должна равняться нулю. 
	
	Теперь дадим несколько определений.
	
\mysubsection{Основные понятия сетей}	
	
\begin{definition}
	Будем называть орграф $D$ \emph{транспортной сетью} (flow network), если задана функция $c\colon E \to R_{+}$, называемая \emph{пропускной способностью} и выделены две точки: \emph{исток $s$} (source) и \emph{сток $t$} (sink)~---~такие, что $indeg(s) = 0$ и $outdeg(t) = 0$.
\end{definition}

	В этом случае под $R_{+}$ подразумеваются неотрицательные действительные числа. Кроме того, чтобы обобществить ещё сильнее понятия, считается, что пропускная способность рёбер, не входящих в $E$, равна нулю.

\begin{definition}
	\emph{Потоком} (flow) называется функция $f$, которая задана на рёбрах (в том числе и не лежащих в $E$) и принимает значения с тремя свойствами для любых вершин $a$ и $b$: 
	$$f(a, b) \leqslant c(a, b), \;\ f(a, b) + f (b, a) = 0, \;\ \sum_{v \in V}f(a, v) = 0.$$
\end{definition}

	Сама терминология обусловлена следующим примером: допустим у нас есть город с развитой дорожной сетью и через него проходит поток машин. Надо сделать так, чтобы через город прошло максимальное число машин при этом ни на одном перекрестке не образовалась пробка.
	
\begin{definition}
	\emph{Величиной потока} (value of flow) называется число 
	$$|f| = \sum_{v \in V} f (s, v) = \sum_{v \in V} f (v, t).$$
\end{definition}

\mysubsection{Алгоритм Форда-Фолкерсона. Теорема Гэйла}
	
	Предположим, что пропускная способность равна либо нулю, либо единице.	
	
	Пусть у нас есть транспортная сеть с каким-то потоком (может, нулевым). Назовём рёбра, по которым пошёл поток, \emph{насыщенными}, а остальные~---~\emph{свободными}.
	
	Далее будем помечать вершины, начиная с истока следующим образом: рассматриваем все смежные с отмеченной вершиной $x$ узлы сети~---~множество $U$. Далее, если есть $u \in U$ такое, что $(x, u)$~---~свободное ребро, то помечаем $x$ <<плюсом>>; если есть $u \in U$ такое, что $(u, x)$~---~насыщенное ребро, то помечаем $x$ <<минусом>>. Повторяем итерацию для новых отмеченных вершин. 
	
	В итоге у нас либо будет помечен сток, либо не будет. В первом случае мы восстанавливаем путь, по которому мы дошли до стока и меняем в нём насыщенные рёбра на свободные и наоборот. Во втором случае~---~говорим, что поток максимален. Докажем это.
	
	\emph{Доказательство.} Обозначим через $U$ (uncolored) множество непомеченных вершин, а через $C$ (colored)множество помеченных. Заметим, что в $C$ входит исток, а в $U$~---~сток. Следовательно, по нашему построению следует, что все рёбра, выходящие из $C$ в $U$ насыщены, а обратные рёбра~---~свободные. Следовательно, в этом случае поток выходит из $C$ и больше не возвращается в него, проходя по всем возможным рёбрам, а значит, это максимальный поток. ч.т.д.
	
	С помощью этого алгоритма можно сформулировать и доказать центральную теорему теории транспортных сетей. 
	
\begin{definition}
	Назовём \emph{пропускной способностью} множества промежуточных вершин $A$ число стрелок, входящих в него и обозначим через $c(A)$.
\end{definition}

\begin{definition}
	Назовём \emph{полной потребностью} множества промежуточных вершин $A$ число стрелок, выходящих из него прямо в сток и обозначим через $d(A)$.
\end{definition}

\begin{theorem}[Гэйла]
	Для того чтобы существовал поток, насыщающий все выходные дуги, необходимо и достаточно, чтобы для любого множества промежуточных вершин $A$ его полная потребность не превосходила пропускной способности: $$d(A) \leqslant c(A).$$
\end{theorem}

%\mysubsection{Задачи}	
%
%\begin{exersize}
%	Докажите теорему Холла, используя теорему Гэйла о насыщенности.
%\end{exersize}	
%\begin{exersize}
%\end{exersize}