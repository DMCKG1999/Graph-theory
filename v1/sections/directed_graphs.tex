\newpage
\section{Об орграфах}

	Ранее мы ничего не говорили про ориентацию графов. Пришло время задеть эту насущную тему. Сейчас мы займёмся вплотную изучением ориентированных графов. Будем их обозначать буквой $D$.
	
\mysubsection{Базовые понятия орграфов}	
	
\begin{definition}
	Будем говорить, что ребро $e$, соединяющее вершины $a$ и $b$, имеет \emph{ориентацию}, подразумевая, что отображение $I$ ставит этому ребру в соответствие упорядоченную пару $(a, b)$ или $(b, a)$. В таком случае ребро также называют \emph{ориентированным}.
\end{definition}

\begin{definition}
	\emph{Орграф}~---~граф, состоящий из ориентированных ребер.
\end{definition}

\begin{definition}
	\emph{Простой орграф}~---~граф, в котором нет петель и кратных упорядоченных рёбер.
\end{definition}

	Заметим, что промежуточного звена между ориентированными и неориентированными графами нет, так как если в любом графе есть хотя бы одно ориентированное ребро, то можно все остальные ребра заменить на два ориентированных противонаправленных ребра и получить орграф, который будет в некотором смысле <<изоморфен>> исходному.
	
	Говоря об изоморфизме орграфов, нельзя уже проверять только наличие и кратность ребер, потому что важно и направление ребер.

	Одним из часто встречаемых орграфов является турнир.

\begin{definition}
	Граф $Y$, полученный из неориентированного графа $G$ путём замены всех рёбер на ориентированные их аналоги, называется \emph{ориентацией графа}.
\end{definition}	
\begin{definition}
	\emph{Турнир $T$}~---~орграф, полученный произвольной ориентацией полного графа $K_n$.
\end{definition}

	Граф $T$ можно встретить при решении задач на круговые турниры, в которых каждый участник играет с каждым и при этом ничьи не бывает. В этом случае ребро обычно направляют от победившего к проигравшему.
\begin{example}
	Сумасшедший король хочет ввести на дорогах своего королевства одностороннее движение так, чтобы выехав из одного города, уже будет нельзя в него вернуться. Удастся ли ему осуществить свою затею?
	
	\emph{Решение.} Рассмотрим сначала частный случай, а именно: полный граф. В таком случае королю удастся осуществить затею, например, следующим образом: выстроим города в ряд и будем все дороги ориентировать слева-направо.
	
	Теперь заметим, что если бы мы имели неполный граф, то это был бы подграф полного. Следовательно, опять же расположим их в ряд и будем все ребра ориентировать слева-направо. В итоге, выходя из какого-то города мы всегда будем попадать в гороод правее него, а значит, никогда не сможем вернуться обратно.
\end{example}

	Рассмотрим некоторые утверждения и понятия, которые мы раньше вводили, но уже на примере орграфа.
	
	Для начала заметим, что так как у нас все рёбра имеют направления, то мы не можем говорить просто про инцидентные рёбра: важно ещё входя они или выходят из вершины.
	
\begin{definition}
	Число рёбер, выходящих из вершины $a$, называется \emph{исходящей степенью вершины} и обозначается $outdeg (a)$. Число рёбер, входящих в вершину $a$, называется \emph{входящей степенью вершины} и обозначается $indeg (a)$.
\end{definition}

\begin{definition}
	Будем говорить, что вершина $b$ \emph{смежна} с вершиной $a$, если есть ребро $(a, b)$, то есть из вершины $a$ можно добраться по ребру до вершины $b$.
\end{definition}

	Можно сформулировать аналог леммы о рукопожатиях.
	
\begin{lemma}[о взятках]
	В любом орграфе имеет место равенство
	$$\sum_i outdeg (x_i) = |E| = \sum_i indeg (x_i).$$
\end{lemma}

	В случае ориентированных графов очень сложно уйти от понятий эвивалентности, так что мы этого не будем делать и сделаем небольшую остановку, чтобы разобраться с ними. 
	
\mysubsection{Отношение эквивалентности}
	
	Для начала рассмотрим произвольное множество $M$. Будем говорить, что на этом множестве задано \emph{бинарное отношение}, если для любой упорядоченной пары $(a, b)$ поставлена в соответствие единица или ноль. Другими словами, мы говорим об отношении, как о правиле, по которому мы объединяем некоторые элементы. Прилагательное <<бинарное>> в этом случае означает, что мы имеем дело с двумя элементами, хотя в общем случае бывают правила, которые задают отношение между большим количеством элементов множества.
	
	Далее есть три свойства отоношения, которые нам понадобятся. Первое из них~---~\emph{рефлексивность}~---~заключается в том, что для пар $(a, a)$ в соответствие должна быть поставлена единица. Суть второго~---~\emph{симметричности}~---~в том, что паре $(a, b)$ поставлена в соответствие единица тогда и только тогда, когда и паре $(b, a)$ поставлена в соответствие единица. Третье из них~---~\emph{транзитивность}~---~уточняет отношение для трёх элементов, а точнее, оно гласит, что если парам $(a, b)$ и $(b, c)$ поставлена в соответствие единица, то и паре $(a, c)$ поставлена в соответствие единица.
	
	Будем говорить, что бинарное отношение является \emph{отношением эквивалентности}, если оно удовлетворяет всем трём свойствам, описанным в предыдущем абзаце.
	
	Например, равенство является отношением эквивалентности. Кроме того, легко проверить, что связность в графе тоже будет таким отношением. 
	
	На этом мучений читателей, впервые столкнувшихся с отношением эквивалентности, не закончены. Надо заметить, что отношение эквивалентности разбивает всё множество на \emph{классы эквивалентности}, внутри которых все объекты эквивалентны. Таким образом, мы можем говорить о \emph{фактормножестве}, суть которого заключается, что мы всему классу эквивалентности ставим в соответствие единственный элемент, то есть буквально отождествляем все эквивалентные элементы.
	
	Используя всю терминологию, связанную с отношением эквивалентности, мы можем сказать, что в неориентированных графах число компонент связности есть ни что иное, как количество классов эквивалентности по отношению связности.

	Теперь вернёмся к орграфам. 	
	
\mysubsection{Связность в орграфах. Конденсат}

\begin{definition}
	Вершины $c$ и $d$ орграфа называются \emph{связанными}, если существует хотя бы один путь из $с$ в $d$ и есть хотя бы один путь из $d$ в $c$.
\end{definition}	 

	Заметим, что при таком определении связность остаётся отношением экививалентности, но в этом случае классы эквивалентности уже называют \emph{сильными компонентами связности}.
	
\begin{definition}
	Орграф называется \emph{сильно связным}, если он состоит из одной сильной компоненты связности. 
\end{definition}

	В случае, если нас интересует связность неориентированного графа, ориентацией которого является исходный граф, то можно пользоваться следуйющим понятием.

\begin{definition}
	Орграф $D$ называется \emph{слабо связным}, если он не является сильно связным, но граф $G$, ориентацией которого является $D$, связен.
\end{definition}

	Как можно заметить, фактор-граф по отношению связности в случае неориентированного графа является всегда набором изолированных точек. Однако при переходе к орграфам всё резко меняется.

\begin{definition}
	Фактор-граф графа $G$ по отношению связности называется \emph{конденсатом} и обозначается $C(G)$.
\end{definition}

\begin{theorem}
	Конденсат любого орграфа $D$~---~ациклический орграф (DAG~---~directed acyclic graph).
	
	\emph{Доказательство.} Допустим, что в нём есть цикл. Тогда между любыми двумя вершинами $C(D)$ в этом цикле есть путь в обе стороны, то есть они связаны, следовательно, должны быть в одной компоненте сильной связности. Противоречие.
\end{theorem}

	Конденсат орграфа занимает центральное место в теории ориентированных графов. С одной стороны, его достаточно легко найти в произвольном графе, а алгоритмы, которые помогают это сделать, требуют мало времени и памяти для работы. С другой стороны, ориентированные ациклические графы имеют широкое практическое применение. 
	
	Например, предположим, что вам надо составить свой курс. У вас есть набор тем, которые ссылаются друг на друга. Построив конденсат графа, в котором вершинами будут темы, а рёбрами будут ссылки, вы сможете понять, какие темы и в каком порядке надо давать.

	Возвращаясь немного назад, скажем ещё пару слов о турнирах.
	
\mysubsection{Транзитивные турниры}	
	
\begin{definition}
	Турнир $T$ называется \emph{транзитивным}, если из условий $(a, b) \in E(T)$, $(b, c) \in E(T)$ следует, что $(a, c) \in E(T)$.
\end{definition}

\begin{statement}
	Турнир является транзитивным, когда он совпадает со свои конденсатом.
	
	\emph{Доказательство.} Это условие эквивалентно тому, что в графе нет циклов. Следовательно, так как все рёбра есть, а циклов нет, то при существовании рёбер $(a, b)$ и $(b, c)$ однозначно восстанавливается ребро между вершинами $a$ и $c$~---~$e = (a, c)$. По определению этот турнир будет транзитивным. ч.т.д.
\end{statement}

\begin{consequence}
	В нетранзитивном турнире есть ориентированный цикл длины 3.
\end{consequence}

\begin{consequence}
	Конденсат турнира всегда является транзитивным турниром.
	
	\emph{Доказательство.} Заметим, что $C(C(T)) = C(T)$.
\end{consequence}

$ $
\newline
	Мы познакомились ближе с орграфами. Узнали про аналогичные неориентированным графам утверждения, термины. А также забежали немного вперёд по программе и поговорили об отношении эквивалентности.

\subsection*{\begin{center}Задачи\end{center}}
\begin{exersize}
	Сколько различных ориентированных графов можно получить из одного и того же простого графа $G(V, E)$ ориентацией его рёбер?
\end{exersize}

\begin{exersize}
	Докажите, что на рёбрах любого связного графа можно расставить стрелки, что найдется вершина, из которой можно было бы добраться по стрелкам в любую другую.
\end{exersize}

\begin{exersize}
	 В некоторой стране каждый город соединен с каждым дорогой с односторонним движением. a) Докажите, что найдется город, из которого можно попасть в любой другой. b) Докажите, что можно поменять направление движения на одной дороге так, что из любого города можно будет попасть в любой другой.
\end{exersize}

\begin{exersize}
	Докажите, что для любого натурального нечётного $n = 2k + 1$ существует турнир, в котором для всех вершин верно равенство
	$$outdeg (x) = k = indeg (x).$$
\end{exersize}

\begin{exersize}
	Докажите, что для любого турнира сумма квадратов входящих степеней всех вершин равна сумме квадратов выходящих степеней всех вершин.
\end{exersize}