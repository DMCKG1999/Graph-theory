\newpage
\section{О степенях вершин и числе рёбер}

	Основы высшей математики и олимпиадной школьной математики включают в себя такие понятия, как множества и отображения. Мы не будем здесь про них писать, считая, что читатель знаком уже с ними.
	
\mysubsection{Простой граф, псевдограф, мультиграф}	
	
	В первую очередь, дадим теперь уже совершенно строгое определение графа.

\begin{definition}
	\emph{Граф}~---~три упорядоченных множества: конечного множество вершин $V$ (от англ. vertex), конечного множество рёбер или дуг $E$ (от англ. edge) и отображение инцидентности $I: E \to V \times V$. Обозначается обычно так $G = \langle V, E, I\rangle$ или $G(V, E)$.
\end{definition}

\begin{paracol}{2}
\begin{example}
	Справа вы можете увидеть представителя выпуклых многоугольников, а именно~---~треугольника. Но нас будет интересовать не геометрическая, а теоретико-графовая конструкция. В терминах графов его можно обозначить так:
	$$G(\lbrace a, b, c\rbrace, \lbrace \lbrace a, b\rbrace, \lbrace b, c\rbrace, \lbrace c, a\rbrace \rbrace).$$
\end{example}

\switchcolumn

\begin{center}\begin{tikzpicture}
	\draw (-0.2, -1.2) node [left] {a}
		  (0.8, 2.2) node [above] {b}
		  (4.2, -0.2) node [right] {c};	

	\tikzstyle{every node}=[circle, draw, fill=red, inner sep=0pt, minimum width=6pt]	
	
	\draw (0, -1) node {} --
		  (1, 2) node {} --
		  (4, 0) node {} -- (0, -1);	
\end{tikzpicture}

	\small Рис. 13. Треугольник
\end{center}

\end{paracol}

	Заметим, что нигде не было сказано, что точки должны располагаться на плоскости. На самом деле граф можно расположить и в пространстве. Например, каркасы любых многогранников представляют из себя графы.

	Уточним определение графа: если под множеством $V \times V$ подразумевается множество упорядоченных пар вершин, то граф называется \emph{ориентированным}. В противном случае он будет \emph{неориентированным}. Впоследствии там, где будут встречаться ориентированные графы, мы их будем обозначать буквой $D$, а в остальных случаях будем пользоваться буквой $G$.

	Для того чтобы было понятно, в каком случае у нас ребро ориентировано, а в каком~---~нет, введём обозначение: $\left\lbrace a, b \right\rbrace$~---~это неориентированное ребро, а $(a, b)$ или $(b, a)$~---~ориентированное.
	
\begin{definition}
	Рёбра называются \emph{кратными}, если они соединяют одни и те же вершины (и в том же порядке для ориентированных рёбер). Также говорят, что вместе кратные рёбра образуют \emph{мультиребро}. Граф, в котором разрешены мультирёбра, называется \emph{мультиграфом}.
\end{definition}

\begin{definition}
	Ребро называется \emph{петлёй} (loop), если оно соединяет вершину саму с собой. Граф, в котором разрешены петли называется \emph{псевдографом}.
\end{definition}

	Эти понятия можно комбинировать, получая по истине удивительные словообразования: псевдомультиграф, мультиорграф, псевдоорграф, псевдомультиорграф.

\begin{example}
	Отображение инцидентности $I$ можно задать в виде таблице, в которой в первой строке будут выписаны рёбра, а во второй~---~соответствующие им пары вершин:

\begin{center}
 \begin{tabular}{||c|c|c|c|c|c|c|c|c||} 
 \hline
 $E$ & $a$ & $b$ & $c$ & $d$ & $e$ & $f$ & $g$ & $h$ \\
 \hline
 $V \times V$ & $\lbrace1, 2\rbrace$ & $\lbrace1, 2\rbrace$ & $\lbrace2, 3\rbrace$ & $\lbrace2, 3\rbrace$ & $\lbrace3, 4\rbrace$ & $\lbrace4, 5\rbrace$ & $\lbrace5, 5\rbrace$ & $\lbrace5, 1\rbrace$ \\ 
 \hline
\end{tabular}\end{center}

	Ей соответствует граф $G$, изображенный на рис. 14.

\begin{center}\begin{tikzpicture}
	\Loop (-2, 2) [dir=WE];	
	
	\node at (0, 0)[below left]{$1$};  
	\node at (4, 0)[below right]{$2$};  
	\node at (4, 4)[above right]{$3$};  
	\node at (0, 4)[above left]{$4$};  
	\node at (-2, 1.8)[below]{$5$};
	
	\node at (2, 0.3){$a$};  
	\node at (2, -0.7){$b$};  
	\node at (3.3, 2){$c$};  
	\node at (4.7, 2){$d$};  
	\node at (2, 4.3){$e$};
	\node at (-1.2, 3.2){$f$};  
	\node at (-4.5, 2){$g$};  
	\node at (-0.8, 1.2){$h$};
	
	\node at (6, 0) {};  
	
	\tikzstyle{every node}=[circle, draw, fill=red, inner sep=0pt, minimum width=6pt]
	[line width=2pt]
	\draw (0, 0) edge [line width=1pt] (4, 0);
	\draw (0, 0) edge [bend right = 25, line width=1pt] (4, 0);
	\draw (4, 0) edge [bend right = 25, line width=1pt] (4, 4);
	\draw (4, 0) edge [bend left = 25, line width=1pt] (4, 4);
	\draw (4, 4) edge [line width=1pt] (0, 4);
	\draw (0, 4) edge [line width=1pt] (-2, 2);
	\draw (-2, 2) edge [line width=1pt] (0, 0);
	
	\node at (0, 0){};  
	\node at (4, 0){};  
	\node at (4, 4){};  
	\node at (0, 4){};  
	\node at (-2, 2){};
\end{tikzpicture}\end{center}
\begin{center}
	\small Рис. 14. Пример псевдомультиграфа 
\end{center}

\end{example}

\begin{paracol}{2}
\begin{definition}
	Неориентированный граф называется \emph{простым}, если он не содержит петель и кратных рёбер.
\end{definition}

	Везде дальше мы будем подразумевать по умолчанию, что работаем только с простыми графами. Из-за этого мы фактически не будем использовать отображение инцидентности, подразумевая, что $E~\subset~V~\times~V$.
	
	Справа, на рис. 15, изображён граф Петерсона, который в своё время стал одним из самых известных примером непланарного графа.
\switchcolumn
\begin{center}\begin{tikzpicture}
	\tikzstyle{every node}=[circle, draw, fill=red, inner sep=0pt, minimum width=6pt]
	\foreach \x in {18,90,...,306}
    {
    	\draw
    	(\x:1.5) node {} -- (\x+144:1.5)
    	(\x:1.5) node {} -- (\x+216:1.5);
    };
    \foreach \x in {18,90,...,306}
    {
    	\draw
    	(\x:2.5) node {} -- (\x:1.5)
    	(\x:2.5) node {} -- (\x+72:2.5);
    };
    \draw \foreach \x in {18,90,...,306}
    {
    	(\x:1.5) node {}
    };
\end{tikzpicture}\end{center}
\begin{center}
	\small Рис. 15. Пример простого графа 
	
	(граф Петерсена)
\end{center}
\end{paracol}

\begin{definition}
	\emph{Полный граф}~---~простой граф, в котором любые две вершины соединены ребром. Обозначение: $K_n$.
\end{definition}

\begin{example}
	Сколько ребер в полном графе $K_n$? Сколько различных графов мы можем получить из него, стирая некоторые ребра?
	
	К ответу на первый вопрос можно прийти несколькими способами. Например, можно заметить, что между парами вершин и ребрами существует взаимно однозначное соответствие, поэтому количество ребер равно $C_n^2 = \frac{n(n-1)}{2}$. Иначе можно заметить, что из каждой вершины выходит ровно $n - 1$ ребро, а всего вершин у нас~---~$n$. Следовательно, $n(n-1)$~---~это удвоенное количество ребер, так как каждое ребро мы посчитали дважды.
	
	Дальше, чтобы понять, сколько мы можем получить различных графов, стирая ребра, заметим, что ребро может быть либо стерто, либо нет. Таким образом, у каждого ребра два возможных состояния. Поэтому количество искомых графов равно $2^{|E|}$, где $|E|$~---~обозначение для количества ребер в графе. 
\end{example}

\mysubsection{Лемма о рукопожатиях}

\begin{definition}
	Пусть $e = \left\lbrace x, y \right\rbrace \in Im_{I}(E)$. Тогда говорят, что вершины $x$ и $y$ \emph{инцидентны} ребру $e$, а ребро $e$ \emph{инцидентно} вершинам $x$ и $y$. Часто также говорят, что $x$, $y$~---~\emph{концевые вершины} ребра $e$. Два ребра называются \emph{смежными}, если они имеют общую вершину. Две вершины называются \emph{смежными}, если они инцидентны одному и тому же ребру.
\end{definition}


\begin{definition}
	Количество ребер, выходящих из данной вершины, называется \emph{степенью} или \emph{валентностью} этой вершины. Подразумевается, что петля даёт двойной вклад. Вершина графа, имеющая нечетную степень называется \emph{нечетной}, а имеющая четную степень~---~\emph{четной}. Далее степень вершины $a$ будем обозначать через $deg(a)$.
\end{definition}

	В примере выше мы уже суммировали степени вершин полного графа. Оказывается, что это не безуспешное занятие и в общем случае.

\begin{lemma*}[о рукопожатиях]
	Сумма степеней вершин произвольного графа равна удвоенному количеству его ребер.
	
	\emph{Доказательство.} Вообщем-то частично мы уже доказали это утверждение выше. Для полного доказательства заметим, что если мы просуммируем степени всех вершин, то каждое ребро мы учтём дважды, так как у каждого ребра ровно два конца. Таким образом, мы и приходим к следующей формуле $$\sum_{a \in V} deg(a) = 2|E|.$$
\end{lemma*}

\begin{consequence}
	Количество нечетных вершин любого графа четно.
\end{consequence}

	Последнее утверждение имеет название <<Лемма о рукопожатиях>>, так как если взять произвольную группу людей, тогда среди них будет всегда четное количество людей, которые в течение дня жали руку нечетному количеству людей из этой же группы. Или, например, за всю историю человечества количество людей, которые жали руку нечётное число раз, чётно.

	Несмотря на вполне тривиальную формулировку и доказательство, эта лемма представляет из себя применение одного из сильных методов олимпиадной математики~---~ двойного подсчёта (double counting). Смысл этого метода состоит в том, что произвольное множество $M$ можно разбить на непересекающиеся подмножества неединственным образом. Например, допустим, что у нас есть два разбиения 
$$M = \bigcup_i A_i, \forall \!\ k \neq m \colon \!\ A_k \cap A_m = \varnothing; M = \bigcup_j B_j,  \forall \!\ k \neq m \colon \!\ B_k \cap B_m = \varnothing.$$

	Тогда можно два раза подсчитать, сколько элементов в $M$
$$\sum_i |A_i| = |M| = \sum_j |B_j|.$$	
	
	В это равенство и заложена суть метода двойного подсчёта.
	
	Добавим, что если читатель~---~школьник, который хочет завоёвывать дипломы на олимпиадах разного уровня, то ему стоит обратить внимание на метод двойного подсчёта, так как он часто встречается в олимпиадах.
	
\begin{example}
	На экскурсию пришло 19 школьников. Могло ли так получится, что каждая девочка знакома ровно с пятью школьниками, а каждый мальчик~---~ровно с тремя школьниками?\\
	
	\emph{Решение.} Заметим, что если школьников обозначить вершинами, а знакомства~---~ребрами, то все вершины окажутся нечётными, но их нечётное число. Противоречие. Следовательно, так не могло быть.
\end{example}

	Целый пласт простых олимпиадных задач решается с помощью применения леммы о рукопожатиях, хотя более экзотическое её применение можно найти только в совокупности с другими более сильными теоремами. 

\mysubsection{Степенные последовательности}

\begin{definition}
	Назовём последовательность $(d_1, d_2, \dots, d_n)$ \emph{правильной}, если $$n-1 \geqslant d_1 \geqslant d_2 \geqslant \dots \geqslant d_n \;\ \text{и} \;\ \sum_{i=0}^n d_i - \!\ \text{чётное число}.$$  
\end{definition}

\begin{definition}
	\emph{Степенная последовательность}~---~это последовательность степеней вершин псевдомультиграфа $G$, записанная в порядке невозрастания. \emph{Графическая последовательность}~---~это правильная последовательность, соответствующая какому-то простому графу.
\end{definition}

	На самом деле тут мы немного схитрили, потому что очевидным образом из леммы о рукопожатиях следует второе условие правильности последовательности, а первое условие~---~из того, что все исходящие рёбра соединяют вершину с какой-то ещё, то есть их меньше, чем $C_{n-1}^1 = n-1$. 
	
	Степенные и графические последовательности начали изучать в связи с поиском универсальной величины, по которой мы смогли бы отличать графы друг от друга. Но, к сожалению, несмотря на критерий, который мы обсудим ниже, этот путь не привёл ни к каким грандиозным результатам.	
	
\begin{statement}[критерий степенной последовательности]
	Последовательность неотрицательных невозрастающих целых чисел $(a_1, a_2, \dots, a_n)$ будет степенной тогда и только тогда, когда сумма чисел в ней чётна.
	
\begin{proof}
	Необходимость чётности следует из леммы о рукопожатиях. Чтобы показать, что этого условия достаточно, построим такой граф.
	
	Во-первых, возьмём $n$ изолированных вершин и пронумеруем их. Во-вторых, проведём $\lfloor \frac{a_i}{2} \rfloor$ петель у $i$-й вершины, и вычтем из всех $a_i$-х удвоенное число соответствующих петель. Пусть после вычитания получились числа $(b_1, b_2, \dots, b_n)$; каждое из них будет равно либо нулю, либо единице. Однако так как мы вычитали чётные числа, то их сумма осталась чётна. Следовательно, можно все вершины разбить на пары и соединить ребром в каждой паре. В итоге мы получили граф, у которого степенная последовательность совпадает с исходной.
\end{proof}\end{statement}

	Таким образом, мы можем легко отделить из всех неотрицательных невозрастающих последовательностей те, которые будут степенными. Однако в случае с графическими последовательностями так легко мы уже не отделаемся.

\begin{theorem}[Эрдеша-Галлаи]
	Правильная последовательность $(d_1, d_2, \dots, d_n)$ является графической тогда и только тогда, когда 
	$$\forall \!\ k \in \BN: 1 \leqslant k \leqslant n-1 \mapsto \sum_{i=1}^k d_i \leqslant k(k-1) + \sum_{i=k+1}^n min \lbrace k, d_i \rbrace. $$
\end{theorem}

	Доказательство этой теоремы мы здесь приводить не будем, потому что материал, который нужен для этого, нельзя разместить в одну-две страницы, поэтому если бы мы поступили иначе, то его пришлось бы нещадно сжимать, что привело бы к потере качества.

	Хочется добавить, что есть даже конструктивный подход, а именно~---~алгоритм Гавела-Хакими, который позволяет по произвольной графической последовательности построить граф, который будет иметь такую степенную последовательность.

	К сожалению, несмотря на такие сильные утверждения, как теорема Эрдеша-Галлаи, графическая последовательность не стала ключом к подсчёту графов, потому что у разных графов она может совпадать, как в следующем примере.
	
\begin{example}
	Нарисовать три различных графа с графической последовательностью $(3, 2, 2, 1, 1, 1)$.
	
\begin{center}\begin{tikzpicture}
	\tikzstyle{every node}=[circle, draw, fill=red, inner sep=0pt, minimum width=6pt]	
	
	\draw (0, 0) node {} -- (1, 1) node {} -- (2, 1) node {};
	\draw (0, 0) node {} -- (1, 0) node {};
	\draw (0, 0) node {} -- (1, -1) node {} -- (2, -1) node {};	
\end{tikzpicture}
\;\ \;\ \;\ \;\ \;\ \;\ \;\ \;\ \;\ \;\ 
\begin{tikzpicture}
	\tikzstyle{every node}=[circle, draw, fill=red, inner sep=0pt, minimum width=6pt]	
	
	\draw (0, 0) node {} -- (1, 1) node {} -- (-1, 1) node {} -- (0, 0) node {} -- (0, -1) node {};
	\draw (1, -1) node {} -- (2, 0) node {};	
\end{tikzpicture}
\;\ \;\ \;\ \;\ \;\ \;\ \;\ \;\ \;\ \;\ 
\begin{tikzpicture}
	\tikzstyle{every node}=[circle, draw, fill=red, inner sep=0pt, minimum width=6pt]	
	
	\draw (0, 0) node {} -- (1, 0) node {} -- (2, 0) node {} -- (3, 0) node {};
	\draw (0, 0) node {} -- (1, 1) node {};
	\draw (0, 0) node {} -- (1, -1) node {};	
\end{tikzpicture}
\newline
\newline
	\small Рис. 16. Три различных графа с одинаковой степенной последовательностью
\end{center}
\end{example}

Кроме теоремы Эрдеша-Галлаи, есть ещё один критерий того, что правильная последовательность будет графической. Сформулируем его.

\begin{statement}
	Правильная последовательность $s_1 = (s, d_1, d_2, \dots, d_n)$ будет графической тогда и только тогда, когда графической будет последовательность $s_2~=~(d_1 - 1, d_2 - 1, \dots, d_s - 1, d_{s+1}, \dots, d_n)$.
\end{statement}

\begin{definition}
	\emph{Регулярным $k$-графом} называется граф, у которого степени всех вершин равны.
\end{definition}

	Очевидно, что у $k$-регулярного графа степенная последовательность будет состоять из одинаковых чисел, а именно $(k, k, \dots, k)$. Регулярный 3-граф также называют \emph{кубическим}. Легко заметить, что полный граф на $n$ вершинах есть ни что иное, как регулярный $(n-1)$-граф.
	
	Одним из самых удивительных примеров регулярного $k$-графа является $k$-мерный куб. Обозначается он так $Q_k$. Ниже, на рис. 17, изображены наименьшие четыре $k$-кубы.

\begin{center} \;\ \;\ \;\ \;\ \;\ \;\
\begin{tikzpicture}
	\draw (0, -0.7) node {$ $};
	\tikzstyle{every node}=[circle, draw, fill=red, inner sep=0pt, minimum width=6pt]	
	
	\draw (0, 0) node {};	
\end{tikzpicture} \;\ \;\ \;\ \;\ \;\ \;\
\begin{tikzpicture}
	\tikzstyle{every node}=[circle, draw, fill=red, inner sep=0pt, minimum width=6pt]	
	
	\draw (0, 0) node {} -- (0, -1.4) node {};	
\end{tikzpicture} \;\ \;\ \;\ \;\ \;\ \;\
\begin{tikzpicture}
	\tikzstyle{every node}=[circle, draw, fill=red, inner sep=0pt, minimum width=6pt]	
	
	\draw (0, 0) node {} -- (-1.4, 0) node {} -- (-1.4, -1.4) node {} -- (0, -1.4) node {} -- (0, 0) node {};	
\end{tikzpicture} \;\ \;\ \;\ \;\ \;\ \;\
\begin{tikzpicture}
	\tikzstyle{every node}=[circle, draw, fill=red, inner sep=0pt, minimum width=6pt]	
	
	\draw (0,0) -- (0.4, 0.4);
	\draw (-1,0) -- (-0.6, 0.4);
	\draw (0,-1) -- (0.4, -0.6);
	\draw (-1,-1) -- (-0.6, -0.6);
	
	\foreach \x in {0, 0.4}
	{
		\draw (0 + \x, 0 + \x) node {} -- (-1 + \x, 0 + \x) node {} -- (-1 + \x, -1 + \x) node {} -- (0 + \x, -1 + \x) node {} -- (0 + \x, 0 + \x) node {};	
	};
\end{tikzpicture}
\newline
\newline
	\small Рис. 17. Слева-направо: $Q_0$, $Q_1$, $Q_2$, $Q_3$
\end{center}

$ $
\newline
	Итак, в этом параграфе мы определили базовые понятия, с которыми дальше будем активно встречаться на протяжении курса. Кроме того, мы смогли вывести соотношение, связывающее степени вершин и количество рёбер в графе, и получить вполне нетривиальное следствие из него.
	
	Начиная с этого, в конце всех параграфов после заключительных слов будет представлен список задач для самостоятельного решения. Также в разделе <<Решебник>> представлены решения всех задач.
		
\newpage
\mysubsection{Задачи}

\begin{exersize}
	Существует ли граф с $8$-ю вершинами с степенями соответственно равными $2, 3, 3, 4, 4, 5, 5, 5$?
\end{exersize}

\begin{exersize}
	В один из рабочих дней сисадмин Иван Константинович получил в качестве задания~---~соединить проводами $33$ компьютера так, чтобы каждый был соединен либо с одним, либо с тремя компьютерами. Сможет ли он выполнить своё задание?
\end{exersize}

\begin{exersize}
	Студент Василий, гуляя по лесу, начал рассматривать деревья. Он заметил, что каждое дерево переплеталось ветвями ровно с тремя другими. Сколько деревьев было в лесу, если Василий насчитал $300$ пар переплетённых деревьев?
\end{exersize}

\begin{exersize}
	Сколько рёбер в полном графе на $12$ вершинах?
\end{exersize}

\begin{exersize}
	Предположим, что в графе $G$ ровно $17$ вершин. Известно, что сумма степеней всех вершин не меньше $85$. Верно ли, что в этом графе обязательно будет вершина со степенью не меньше шести?
\end{exersize}

\begin{exersize}
	Докажите, что для любых смежных вершин $a$ и $b$ ребро, соединяющее их, будет принадлежать минимально к $deg (a) + deg (b) - |V|$ треугольникам в графе $G(V, E)$.
\end{exersize}

\begin{exersize}
	Пусть $G$~---~простой регулярный связный граф имеющий 34 ребра. Сколько вершин может содержать этот граф?
\end{exersize}

\begin{exersize}
	В сборной по футболу от клуба <<Солянка>> участники~---~это марсиане, земляне и венерианцы. У марсиан по три руки, а у венерианцев по пять. Известно, что в команде $8$ марсиан, $6$ землян и $3$ венерианца. Могут ли они все взяться за руки?
\end{exersize}

\begin{exersize}
	В графе из каждой вершины выходит по $19$ рёбер. Может ли в нём быть $2018$ рёбер? А $2090$?
\end{exersize}

\begin{exersize}
	Докажите, что в простом графе с $n \geqslant 2$ вершинами всегда найдутся хотя бы две вершины с одинаковыми степенями. Останется ли верно это утверждение, если мы будем иметь дело с мультиграфом?
\end{exersize}

\begin{exersize}
	Докажите, что существует граф $G, |V| = 2k$, в котором нет троек вершин с одинаковой степенью. И при этом степень любой вершины не больше $k$ и не равна нулю.
\end{exersize}

\begin{exersize}
	a) Докажите, что нельзя занумеровать рёбра куба числами от $1$ до $12$ так, чтобы для каждой вершины сумма номеров, выходящих из вершины была одной и той же. b) Можно ли убрать одно из чисел от $1$ до $13$, чтобы оставшимися мы смогли сделать искомую нумерацию рёбер?
\end{exersize}

\begin{exersize}
	Докажите, что правильная последовательность, которая выглядит следующим образом
	 $$(a+1, a+1, \dots, a+1, a, \dots, a),$$
	 всегда является графической.
\end{exersize}

\begin{exersize}
	Пусть $G$~---~граф, построенный на вершинах $1, 2, \dots, 15$, в котором вершины $i$ и $j$ смежны тогда и только тогда, когда их наибольший общий делитель больше единицы. Подсчитайте число связных компонент такого графа, а также определите максимальную длину простого пути в графе $G$.
\end{exersize}
