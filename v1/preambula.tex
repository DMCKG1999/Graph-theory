\sloppy
\chapter*{От автора}
\addcontentsline{toc}{chapter}{От автора}

	Теория графов начала своё развитие более $280$ лет назад. За это время элементы этой теории проникли и начали активно использоваться во многих областях, например, связанных с транспортом, с электрическими цепями, теорией информации. Сейчас это одна из самых динамично развивающихся областей комбинаторики.
	
	Кроме того, теория графов не остановилась на фундаментальных науках, а начала появляться в тех областях, в которых её уж точно не ожидали: например, в экономике, в архитектуре, в лингвистике.	
	
	Сейчас теоретико-вероятностные модели, основанные на случайных графах, широко применяются при решении задач, связанных с веб-сетями, например, с Интернетом. В силу широкого распространения теории графов как инструмента для работы с различными сетями за последние полвека эта теория обросла достаточно большим количеством утверждений, теорем. Теперь невозможно работать с ней без знания конечного, но при этом огромного числа терминов. Число новых алгоритмов на графах с каждым годом растёт и уже невозможно представить себе хорошего программиста, который не был бы знаком с алгоритмом Дейкстры, алгоритмом Беллмана-Форда и другими.
	
	Попутно с углублением теории графов она начала активно использоваться при составлении школьных и студенческих олимпиад. В связи с этим сейчас, как никогда прежде, изучение этой области будет полезно для любителей математики всех возрастов. При этом пугает тенденция многих наших вузов сокращать, а не расширять, программу, связанную с теорией графов; в этом смысле зарубежные коллеги пошли правильным путём и число курсов по графам у них только растёт.
	
	В этом курсе почти у всех утверждений написано доказательство, все необходимые определения даны в строгом, формальном виде. Однако несмотря на то что так много внимания уделено формальностям, я постарался разбавить текст примерами, отсылками, интересными фактами, которые смогут сделать приятным чтение этого пособия. Кроме того, в конце каждого конструктивного параграфа читатель может найти задачи для самостоятельного решения.


\begin{flushright}
\textit{Дима Г.}

\textit{\today}
\end{flushright}