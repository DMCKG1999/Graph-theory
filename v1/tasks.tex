\documentclass[12pt,a4paper,fleqn]{article}
\usepackage[utf8]{inputenc}
%\usepackage{amssymb, amsmath, multicol}
\usepackage[russian]{babel}
%\usepackage{concmath}
%\usepackage{euler}
\usepackage{tikz}
\usepackage{paracol}
\usetikzlibrary{arrows}

\oddsidemargin=-15.4mm
\textwidth=190mm
\headheight=-32.4mm
\textheight=277mm
\tolerance=100
\parindent=0pt
\parskip=8pt
\pagestyle{empty}

\newtheorem{task}{Задача}

\begin{document}
\begin{center}
	\bf \Large Графы
\end{center}
\begin{task}
	\emph{Условие.} Назовём доску $N \times M$ с вписанными в каждую клетку числами \textit{красивой}, если нет двух совпадающих строк или столбцов. a) Докажите, что из красивой доски можно вычеркивать столбцы или строки так, чтобы после каждого вычеркивания оставшаяся доска была красивой и чтобы в конце осталась доска $1 \times 1$. b) Приведите пример доски, для которой возможный порядок таких вычеркиваний единственен.
	
	\emph{Доказательство.} Допустим, что $M \geqslant N$ (от этого общность решения не уменьшится) и что нет такого столбца, после вычеркивания которого доска останется красивой. Назовем \textit{соседними} те строки, которые отличаются только в одном элементе. Тогда мы можем утверждать, что есть, как минимум, $M$ различных пар соседних строк.
	
	Почему? Из предположения следует, что после вычеркивания произвольного столбца есть пара совпадающих строк, а так как изначально доска была красивой, то элементы этих строк, лежащие в вычеркнутом столбце различны. Следовательно, эти две строки будут соседними. Таким образом, каждому столбцу можно сопоставить хотя бы одну пару соседних строк. Теперь осталось заметить, что для разных столбцов пары соседних строк будут обязательно различными, потому что иначе единственный элемент, по которому они отличаются, обязан лежать и в одном, и в другом столбце.
	
	Перейдем теперь к интерпретации нашей доски в терминах теории графов: пусть строки будут вершинами, а соседние строки будут соединены ребром. По доказанному выше следует, что в этом графе не меньше $M$ ребер. Так как вершин в нём $N$, а $M > N - 1$, то в этом графе должен быть цикл. 
	
	Пронумеруем все вершины цикла от $1$ до $k$ по направлению обхода. Тогда пусть 
\end{task}
\end{document}