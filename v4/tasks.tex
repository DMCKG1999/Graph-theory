\documentclass[12pt,a4paper,fleqn]{article}
\usepackage[utf8]{inputenc}
\usepackage{amssymb, amsmath, multicol}
\usepackage[russian]{babel}
%\usepackage{concmath}
%\usepackage{euler}
\usepackage{tikz}
\usepackage{paracol}
\usetikzlibrary{arrows}

\oddsidemargin=-15.4mm
\textwidth=190mm
\headheight=-32.4mm
\textheight=277mm
\tolerance=100
\parindent=0pt
\parskip=8pt
\pagestyle{empty}

\newtheorem{task}{Задача}
\newtheorem{exersize}{Задача}

\begin{document}
\begin{center}
	\bf \Large Графы
\end{center}
\begin{task}
	\emph{Условие.} Назовём доску $N \times M$ с вписанными в каждую клетку числами \textit{красивой}, если нет двух совпадающих строк или столбцов. a) Докажите, что из красивой доски можно вычеркивать столбцы или строки так, чтобы после каждого вычеркивания оставшаяся доска была красивой и чтобы в конце осталась доска $1 \times 1$. b) Приведите пример доски, для которой возможный порядок таких вычеркиваний единственен.
	
	\emph{Доказательство.} Допустим, что $M \geqslant N$ (от этого общность решения не уменьшится) и что нет такого столбца, после вычеркивания которого доска останется красивой. Назовем \textit{соседними} те строки, которые отличаются только в одном элементе. Тогда мы можем утверждать, что есть, как минимум, $M$ различных пар соседних строк.
	
	Почему? Из предположения следует, что после вычеркивания произвольного столбца есть пара совпадающих строк, а так как изначально доска была красивой, то элементы этих строк, лежащие в вычеркнутом столбце различны. Следовательно, эти две строки будут соседними. Таким образом, каждому столбцу можно сопоставить хотя бы одну пару соседних строк. Теперь осталось заметить, что для разных столбцов пары соседних строк будут обязательно различными, потому что иначе единственный элемент, по которому они отличаются, обязан лежать и в одном, и в другом столбце.
	
	Перейдем теперь к интерпретации нашей доски в терминах теории графов: пусть строки будут вершинами, а соседние строки будут соединены ребром. По доказанному выше следует, что в этом графе не меньше $M$ ребер. Так как вершин в нём $N$, а $M > N - 1$, то в этом графе должен быть цикл. 
	
	Пронумеруем все вершины цикла от $1$ до $k$ по направлению обхода. Тогда пусть 
\end{task}

\begin{exersize}[Канель-Белов А.Я., Ковальджи А.К. Как решают нестандартные задачи]
	$50$ клеток шахматной доски и $50$ шашек занумерованы числами от $1$ до $50$. Шашки поставили на доску произвольным образом. За один ход любую шашку можно поставить на любое свободное место. Докажите, что для расстановки шашек на свои места требуется не больше $75$ ходов.
\end{exersize}

\begin{exersize}[Федоров Р. М., Канель-Белов А. Я., Ковальджи А. К., Ященко И. В. <<Московские математические олимпиады>>]
	Каждый из $1994$ депутатов парламента дал пощечину ровно одному своему коллеге. Докажите, что можно составить парламентскую комиссию из $665$ человек, члены которой не выяснили отношений между собой указанным выше способом.
\end{exersize}	

\begin{exersize}[Бабичева Т.С., Бабичев С.Л., Жогов А. А., Яковлев И.В. <<Пособие по олимпиадной математике. Уровень А1>>]
	В компании у каждых двух людей ровно пять общих знакомых. Докажите, что количество пар знакомых делится на 3.
\end{exersize}

\begin{exersize}[Агаханов Н.Х., Богданов И.И., Кожевников П.А., Подлипский О.К., Терешин Д.А. <<Всероссийские олимпиады школьников по математике 1993~---~2009: Заключительные этапы>>]
	В один из дней оказалось, что каждый житель города сделал не более одного звонка по телефону. 
	Докажите, что население города можно разбить не более, чем на три группы так, чтобы жители, входящие в одну группу, 
	не разговаривали в этот день между собой по телефону.
\end{exersize}	 

\begin{exersize}[<<Квант>>, 1989, 6 выпуск, М1168]
	В стране $1988$ городов и $4000$ дорог. Докажите, что можно указать кольцевой маршрут, проходящий не более, 
	чем через $20$ городов (каждая дорога соединяет два города).
\end{exersize}

\begin{exersize}[Омельченко А. В. <<Теория графов>>]
	Докажите, что для любых смежных вершин $a$ и $b$ ребро, соединяющее их, будет принадлежать минимально к 
	$deg (a) + deg (b) - |V|$ треугольникам в графе $G(V, E)$.
\end{exersize}

\begin{exersize}[Агаханов Н.Х., Богданов И.И., Кожевников П.А., Подлипский О.К., Терешин Д.А. <<Всероссийские олимпиады школьников по математике 1993~---~2009: Заключительные этапы>>]
	В стране $N$ $1988$ городов и из каждого осуществляются беспосадочные перелеты в три других города (все авиарейсы двусторонние). Известно, что из любого города, сделав несколько пересадок, можно долететь до любого другого. Министр Безопасности хочет объявить закрытыми $200$ городов, никакие два из которых не соединены авиалинией. Докажите, что это можно сделать так, чтобы можно было долететь из любого незакрытого города в любой другой, не делая пересадок в закрытых городах.
\end{exersize}	 

\begin{exersize}[Агаханов Н.Х., Богданов И.И., Кожевников П.А., Подлипский О.К., Терешин Д.А. <<Всероссийские олимпиады школьников по математике 1993~---~2009: Заключительные этапы>>]
	$25$ мальчиков и несколько девочек собрались на вечеринке и обнаружили забавную закономерность. Если выбрать любую группу не меньше чем из $10$ мальчиков, а потом добавить к ним всех девочек, знакомых хотя бы с одним из этих мальчиков, то в получившейся группе число мальчиков окажется на $1$ меньше, чем число девочек. Докажите, что некоторая девочка знакома не менее чем с $16$ мальчиками.
\end{exersize}	 

\begin{exersize}[Агаханов Н.Х., Богданов И.И., Кожевников П.А., Подлипский О.К., Терешин Д.А. <<Всероссийские олимпиады школьников по математике 1993~---~2009: Заключительные этапы>>]
	В некоторой группе из $12$ человек среди каждых $9$ найдутся $5$ попарно знакомых. Докажите, что в этой группе найдутся $6$ попарно знакомых.
\end{exersize}	 

\begin{exersize}[Агаханов Н.Х., Богданов И.И., Кожевников П.А., Подлипский О.К., Терешин Д.А. <<Всероссийские олимпиады школьников по математике 1993~---~2009: Заключительные этапы>>]
	В лагерь приехало несколько пионеров, каждый из них имеет от $50$ до $100$ знакомых среди остальных. Докажите, что пионерам можно выдать пилотки, покрашенные в $1331$ цвет так, чтобы у знакомых каждого пионера были пилотки хотя бы $20$ различных цветов.
\end{exersize}	 


\begin{exersize}[Агаханов Н.Х., Богданов И.И., Кожевников П.А., Подлипский О.К., Терешин Д.А. <<Всероссийские олимпиады школьников по математике 1993~---~2009: Заключительные этапы>>]
	В стране несколько городов, некоторые пары городов соединены дорогами, причем между любыми двумя городами существует единственный несамопересекающийся путь по дорогам. Известно, что в стране ровно $100$ городов, из которых выходит по одной дороге. Докажите, что можно построить $50$ новых дорог так, что после этого даже при закрытии любой дороги можно будет из любого города попасть в любой другой.
\end{exersize}	 

\begin{exersize}[Агаханов Н.Х., Богданов И.И., Кожевников П.А., Подлипский О.К., Терешин Д.А. <<Всероссийские олимпиады школьников по математике 1993~---~2009: Заключительные этапы>>]
	В стране $100$ городов, некоторые пары городов соединены дорогами. Для любых четырех городов существуют хотя бы две дороги между ними. Известно, что не существует маршрута, проходящего по каждому городу ровно один раз. Докажите, что можно выбрать два города таким образом, чтобы любой из оставшихся городов был соединен дорогой хотя бы с одним из двух выбранных городов.
\end{exersize}	

\begin{exersize}[Агаханов Н.Х., Богданов И.И., Кожевников П.А., Подлипский О.К., Терешин Д.А. <<Всероссийские олимпиады школьников по математике 1993~---~2009: Заключительные этапы>>]
	В стране несколько городов, некоторые пары городов соединены дорогами. При этом из каждого города выходит хотя бы $3$ дороги. Докажите, что существует циклический маршрут, длина которого не делится на $3$.
\end{exersize}	 

\end{document}