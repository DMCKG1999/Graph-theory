\documentclass[12pt,a4paper,fleqn]{article}
\usepackage[utf8]{inputenc}
\usepackage{amssymb, amsthm, multicol}
\usepackage[russian]{babel}
\usepackage{amsmath}
\usepackage{concmath}
\usepackage{euler}
\usepackage{tikz}
\usepackage{paracol}
\usetikzlibrary{arrows}

\oddsidemargin=-15.4mm
\textwidth=190mm
\headheight=-32.4mm
\textheight=277mm
\tolerance=100
\parindent=0pt
\parskip=8pt
\pagestyle{empty}

\theoremstyle{definition}
\newtheorem{thm}{Theorem}[section]
\newtheorem{lem}[thm]{Lemma}
\newtheorem{exersize}{Задача}[section]

\begin{document}
\section{Основные понятия и определения (д/з)}
\begin{exersize}
	Приведите как можно больше прилагательных, которыми можно описать граф $K_5$ (например, неориентированный или ориентированный, связный или несвязный). Приведите еще один пример графа, к которому подошли бы все эти эпитеты.
\end{exersize}

\begin{exersize}[Деревья]
	На каникулах братья Володя и Никита от скуки придумали следующую игру. У них на стене весела карта России с отмеченными на ней железными дорогами. За ход разрешалось взять маркер и <<перерезать>> одну из ж/д дорог, но только в том случае, если города, которые соединяет эта дорога не потеряют от этого сообщение между друг другом. Проигрывал тот, кто не мог больше сделать ход. Докажите, что, посмотрев на изначальную карту России, можно сказать еще до окончания игры, кто выиграет.
\end{exersize}

\begin{exersize}[Эйлеров граф]
	Москвич Василий Петрович, приехавший в Санкт-Петербург поездом, весь день ходил по городу пешком. После столь утомительной прогулки он решил <<повысить градус>>. Посмотрев на карту, Василий Петрович заметил, что на всех улицах, по которым он проходил нечетное количество раз, расположены питейные заведения. Этот факт не мог не обрадовать москвича, так что было принято решение: вернуться на вокзал, проходя только по этим улицам. Докажите, что у Василия Петровича всё получится.
\end{exersize}
\begin{exersize}[Связный граф]
	В группе из нескольких человек некоторые люди знакомы друг с другом, а некоторые нет. Каждый вечер один из них устраивает ужин для всех своих знакомых, на котором знакомит их друг с другом. После того, как каждый человек устроил хотя бы по одному ужину, оказалось, что какие-то два человека все еще не знакомы. Докажите, что они не познакомятся и на следующем ужине.
\end{exersize}

\newpage
\section{Деревья (д/з)}
\begin{exersize}
	Докажите, что граф является деревом тогда и только тогда, когда каждые две его вершины соединены ровно одним путем с различными ребрами.
\end{exersize}	
\begin{exersize}
	a) Ребра дерева окрашены в два цвета. Если в какой-то вершине сходятся ребра одного цвета, то можно их все перекрасить в другой цвет. Можно ли все дерево сделать одноцветным?
	
	b) Будем красить в два цвета не ребра, а вершины графа. Можно ли любое дерево раскрасить так, что любое ребро будет соединять вершины разных цветов?
	
	c) Докажите, что вершины графа можно раскрасить в два цвета тогда и только тогда, когда граф не содержит циклов нечетной длины.
\end{exersize}	
\begin{exersize}
	В течение предвыборной кампании каждый из 600 чиновников от партии <<Ядро>> давал взятку ровно одному своему коллеге. Докажите, что после переизбрания президента можно на 200 государственных мест назначить чиновников из этой партии так, что среди выбранных чиновников никто никому не давал взятку.
\end{exersize}	
\begin{exersize}
	В стране 27 городов, некоторые из них соединены авиалиниями, принадлежащими трём авиакомпаниям. Известно, что даже если любая из авиакомпаний прекратит полёты, можно будет добраться из любого города в любой другой (возможно, с пересадками), пользуясь рейсами оставшихся двух компаний. Какое наименьшее число авиалиний может быть в стране?
\end{exersize}	
\begin{exersize}
	Докажите, что в любом связном графе можно удалить вершину вместе со всеми выходящим ребрами так, чтобы он остался связным.
\end{exersize}	
\begin{exersize}
	Расстоянием между двумя произвольными вершинами дерева будем называть длину простого пути, соединяющего их. Удаленностью вершины дерева назовём сумму расстояний от неё до всех остальных вершин. Докажите, что в дереве, у которого есть две вершины с удаленностями, отличающимичся на 1,~---~нечётное число вершин.
\end{exersize}	
\begin{exersize}
	При каких $n$ число неизоморфных попарно деревьев на $n$ вершинах будет равно $n$?
\end{exersize}

\newpage
\section{Связные графы. Изоморфизм (д/з)}
\begin{exersize}
	Верно ли, что при любом натуральном $k$, если в графе ровно $4k$ вершин имеют степень 5, а степени остальных~---~6, то нельзя удалить одно ребро так, чтобы этот граф распался на две изоморфные компоненты связности?
\end{exersize}
\begin{exersize}
	Дан лес с $k$ компонентами связности и $n$ вершинами. Сколько в нём ребер?
\end{exersize}

\newpage
\section{Планарные графы (д/з)}
\begin{exersize}
	В некоторой стране есть $n$ озер, которые соединены $k$ каналами. Из любого озера по каналам можно добраться в любое другое озеро. Сколько в этой стране островов?
\end{exersize}
\begin{exersize}
	Докажите, что для связного плоского графа выполняются неравенства:
	$$a) 2|E| \geqslant 3|F| \;\ \text{при} |V| \geqslant 2; b) |E| \leqslant 3|V| - 6 \text{при} |V| \geqslant 3.$$
\end{exersize}
\begin{exersize}
	На плоскости отмечено несколько точек, никакие три из которых не лежат на одноц прямой. Двое по очереди соединяют отрезком две какие угодно еще не соединенные точки так, чтобы отрезки не пересекались нигде, кроме отмеченных точек. Проигрывает тот, кто не сможет сделать ход. Докажите, что один из играющих будет всегда выигрывать независимо от своей игры и игры соперника.
\end{exersize}
\begin{exersize}
	Один из простейших многоклеточных организмов~---~водоросль <<вольвокс>>~---~представляет собой сферическую оболочку,, сложенную семиугольными, шестиугольными и пятиугольными клетками (в каждой <<вершине>> сходятся три клетки). Биологи заметили, что пятиугольных клеток всегда ровно на 12 больше, чем семиугольных (всего клеток может быть несколько сотен и даже тысяч). Не можете ли вы объяснить этот странный факт?
\end{exersize}
\begin{exersize}
	Пусть все грани выпуклого многогранника~---~правильные $n$-угольники, и в каждой его вершине сходится ровно $k$ граней. Докажите, что тогда $\frac{1}{n} + \frac{1}{k} = \frac{1}{2} + \frac{1}{r}$, где $r$~---~число его рёбер.
\end{exersize}


\newpage
\section{Эйлеровы и гамильтоновы циклы (д/з)}
\begin{exersize}
	При каких $n$ граф $K_n$ будет эйлеровым?
\end{exersize}
\begin{exersize}
	Верно ли, что можно на доске нарисовать произвольное количество касающихся окружностей, не отрывая мела? (если да, то докажите, что при произвольном расположении получится; если нет, то приведите контрпример)
\end{exersize}

\newpage
\section{Ориентированные графы (д/з)}
\begin{exersize}
	Докажите, что на рёбрах любого связного графа можно расставить стрелки, что найдется вершина, из которой можно было бы добраться по стрелкам в любую другую.
\end{exersize}
\begin{exersize}
	 В некоторой стране каждый город соединен с каждым дорогой с односторонним движением. a) Докажите, что найдется город, из которого можно попасть в любой другой. b) Докажите, что можно поменять направление движения на одной дороге так, что из любого города можно будет попасть в любой другой.
\end{exersize}

\newpage
\section{Двудольные графы. Паросочетания (д/з)}

\newpage
\section{Сети и потоки (д/з)}

\newpage
\section{Раскраска графов (д/з)}

\end{document}