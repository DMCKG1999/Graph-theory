\documentclass[12pt,a4paper,fleqn]{article}
\usepackage[utf8]{inputenc}
\usepackage{amssymb, amsmath, multicol}
\usepackage[russian]{babel}
\usepackage{concmath}
\usepackage{euler}
\usepackage{tikz}
\usepackage{paracol}
\usetikzlibrary{arrows}

\oddsidemargin=-15.4mm
\textwidth=190mm
\headheight=-32.4mm
\textheight=277mm
\tolerance=100
\parindent=0pt
\parskip=8pt
\pagestyle{empty}

\newtheorem{exersize}{Задача}

\begin{document}
\begin{center}
	\bf \Large ДОМАШНЕЕ ЗАДАНИЕ - 2
	
	\bf \Large ТЕОРИЯ ГРАФОВ (PHYSTECH.INTERNATIONAL)
\end{center}

\setcounter{exersize}{20}

\begin{exersize}
	В норке живет семья из $24$ мышей. Каждую ночь ровно четыре из них отправляются на склад за сыром. 
	Может ли так получиться, что в некоторый момент времени каждая мышка побывала на складе с каждой ровно по одному разу?
\end{exersize}

\begin{exersize}
	В общежитии живут $214$ студентов. Каждый час ровно $4$ из них отправляются на кухню перекусить. 
	Может ли так получится, что в некоторый момент времени каждый из студентов столкнулся с каждым на кухне ровно по одному разу?
\end{exersize}
	
\begin{exersize}
	Дано дерево с $n \geqslant 2$ вершинами. В каждую вершину поставили по действительному числу $x_1, x_2, \dots, x_n$. Далее на каждом ребре записали произведение чисел в концевых вершинах. Сумма всех чисел на ребрах получилась равной $S$. Докажите, что имеет место следующее неравенство $$\sqrt{n-1} (x_1^2 + x_2^2 + \dots + x_n^2) \geqslant 2S.$$
\end{exersize}	

\begin{exersize}
	Докажите, что нельзя так раскрасить кубик с гранью в одну клетку в черный и белый цвета, чтобы его можно было прокатить по доске и он побывал на каждой клетке единожды и каждый раз соприкасающаяся с доской грань кубика и клетка, на которой он стоит, были одного цвета.
\end{exersize}	

\begin{exersize}
	Множество клеток на клетчатой плоскости назовем \emph{ладейно связным}, если из любой его клетки можно попасть в любую другую, 
	двигаясь по клеткам этого множества ходом ладьи (ладье разрешается перелетать через поля, не принадлежащие нашему множеству). 
	Докажите, что ладейно связаное множество из $100$ клеток можно разбить на пары клеток, лежащих в одной строке или в одном столбце.
\end{exersize}	

\begin{exersize}
	В некоторой группе из $12$ человек среди каждых $9$ найдутся $5$ попарно знакомых. Докажите, что в этой группе найдутся $6$ попарно знакомых.
\end{exersize}	 

\begin{exersize}
	В стране $N$ городов. Между любыми двумя из них проложена либо автомобильная, либо железная дорога. Турист хочет объехать страну, побывав в каждом городе ровно один раз, и вернуться в город, с которого он начнет путешествие, и маршрут так, что ему придется поменять вид транспорта не более одного раза.
\end{exersize}	 

\end{document}