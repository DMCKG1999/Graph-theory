\section{Линейная реализация графов}

\mysubsection{Введение. Изображение на плоскости}



\begin{definition}
	Фигура $F \in \BR^2$ называется \emph{кусочно-линейной реализацией графа $G$ на плоскости}, если она состоит из набора точек $\lbrace x_k \rbrace_{k=1}^{|V|}$, которому можно сопоставить в соответствие вершины графа ($\exists \!\ \varphi \colon x_k \mapsto v_k$) так, чтобы вершины $v_i$ и $v_j$ смежны тогда и только тогда, когда существует ломаная, начинающаяся в точке $x_i = \varphi^{-1}(v_i)$, заканчивающаяся в точке $x_j = \varphi^{-1}(v_j)$ и не проходящая через другие точки из набора.
\end{definition}

	Такая реализация графа на плоскости, что все ломаные, соединяющие вершины, отрезки, называется \emph{линейной}. Следующая теорема доказывает эквивалентность обоих реализаций на плоскости. Однако предварительно нужно дать несколько определений, чтобы сократить доказательство до приемлемых размеров.
	
\mysubsection{Теорема Фари-Вагнера}	
	
	Само доказательство теоремы будет устроено индуктивно, но в нем мы будем пользоваться определениями, перечисленными ниже.	
	
\begin{theorem}[Фари-Вагнера]
	Граф $G$ линейно реализуем на плоскости $\Leftrightarrow$ он кусочно-линейно реализуем на плоскости $\Leftrightarrow$ $G$ планарен.
\end{theorem}  


\mysubsection{Примеры}

\begin{example}
\end{example}