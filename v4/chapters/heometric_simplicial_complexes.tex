\section{Геометрические симплициальные комплексы}

\begin{definition}
	Будем говорить, что $x_0, \dots, x_k \in \BR$ \emph{аффинно зависимы}, если существует такой набор $\lbrace \alpha_i \rbrace_{i = 0}^{k}$, в котором есть хотя бы один ненулевой элемент, такой, что $\sum_{i = 0}^{k} \alpha_i x_i = 0$ и $\sum_{i=0}^{k} \alpha_i = 0$. В противном случае этот набор точек называется \emph{аффинно независимым}.
\end{definition}

\begin{lemma}
	Имеет место три эквивалентные утверждения для набора точек $x_0, \dots, x_k \in \BR^d$:
\begin{itemize}
	\item этот набор аффинно независим,
	\item вектора $x_1 - x_0, \dots, x_k - x_0$ линейно независимы,
	\item вектора $(1, x_0), \dots, (1, x_k) \in \BR^{d+1}$ линейно независимы.
\end{itemize}
\end{lemma}

\begin{definition}
	\emph{Симплекс $\sigma$}~---~это выпуклая оболочка конечного аффинно независимого множества $A \in \BR^d$. Точки из $A$ называются \emph{вершинами} $\sigma$.  \emph{Размерностью симплекса} называют $dim \sigma \colon = |A| - 1 = k$ и говорят, что это $k$-симплекс.
\end{definition}

\begin{definition}
	\emph{Подсимплекс}~---~это выпуклая оболочка конечного подмножества вершин симплекса $\sigma$. \emph{Относительная внутренность} симплекса $\sigma$ получается из него выкидыванием всех подсимплексов меньшего размера.
\end{definition}

\begin{definition}
	\emph{Симплициальный комплекс}~---~это семейство $\Delta$ симплексов, которые удовлетворяют двум условиям:
\begin{itemize}
	\item подсимплекс любого симплекса $\sigma \in \Delta$ также симплекс из этого семейства,
	\item пересечение любых двух симплексов семесйтв является подсимплексом каждого из них.
\end{itemize}
\end{definition}

	В случае, когда $\Delta$ является набором всех подсимплексов какого-то симплекса $\sigma$, то его называют \emph{многогранником}, притом в нем для каждого $x \in \Delta$ есть ровно один симплекс, в относительной внутренности которого лежит этот $x$. Эту относительную внутренность называют \emph{носителем $x$} и пишут $supp(x)$.
	
\begin{definition}
	\emph{Подкомплекс}~---~подмножество симплициального комплекса, которое является комплексом.
\end{definition}

	Важным примером подкомплексом является \emph{$k$-скелет}, который является подкомплексом, содержащим все симплексы размерности не больше $k$.