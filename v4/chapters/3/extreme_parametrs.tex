\section{Экстремальные параметры}

\begin{definition}
	\emph{Инвариант графа $G$} "--- число $\lambda = \lambda (G)$, которое равно на всех изоморфных друг другу графах.
\end{definition}

	Практически с появления теории графов ученые искали нетривиальные \emph{полные наборы инвариантов}, которые однозначно будут задавать граф. Однако до сих пор такого не нашли. В процессе изучения инвариантов было получено очень много интересных утверждений, связывающих граф и его инварианты. Сейчас мы обсудим некоторые из них.

\mysubsection{Треугольники в графе}

	Здесь и далее мы будем говорить только о неориентированных простых графах.

	Вспомним, что для циклов в графе принято общее обозначение $C_n$. Для тех, кто знаком с основами теории групп, отмечу, что сходство в обозначениях с циклическими группами здесь неспроста: действие образующего на саму же группу порождает циклический сдвиг, который можно изобразить в качестве ориентации цикла графа. 
	
	Среди всех циклов первым интересным является $C_3$, который принято называть \emph{треугольником}.
	
\begin{definition}
	\emph{Обхватом графа $G$} называется длина наименьшего цикла $C_n \subset G$. \emph{Окружением графа $G$} называется длина наибольшего цикла $C_n \subset G$. 
\end{definition}	

\begin{testquestion}
	Чему равен обхват графа, содержащего треугольник?
\end{testquestion}
	
\begin{example}[Омельченко А. В. <<Теория графов>>]
	Рассмотрим произвольную смежную пару вершин $\lbrace a, b\rbrace$ в простом графе $G$ на $n$ вершинах. Докажите, что ребро $e = \lbrace a, b\rbrace$ принадлежит по меньшей мере $deg(a) + deg(b) - n$ треугольникам в графе $G$.

\begin{proof}
	Рассмотрим два множества $U_1(a)$ и $U_1(b)$. Их мощности можно выразить через степени соответствующих вершин $|U_1(a)| = deg(a) + 1$, $|U_1(b)| = deg(b) + 1$.
	
	Используя формулу для мощности пересечения, имеем
	$$|U_1(a)\cap U_1(b)| = |U_1(a)| + |U_1(b)| - |U_1(a) \cup U_1(b)| \geqslant deg(a) + deg(b) + 2 - |V|.$$
	
	Осталось заметить, что любая вершина из пересечения окрестностей вершин дополняет пару $a, b$ до треугольника, так что число треугольников, содержащих $e$ будет не меньше $|U_1(a)\cap U_1(b)|$. Из этого явно следует искомое неравенство. 
\end{proof}
\end{example}

	Большой успех в этом направлении добился Туран. Давайте сначала рассмотрим частный случай, связанный с треугольниками графа, а потом докажем обобщенный вариант.
	
\begin{theorem}
	Наибольшее число ребер у графов, имеющих $p$ вершин и не содержащих треугольников, равно $\lfloor \frac{p^2}{4} \rfloor$. 
\end{theorem}

	Введем в рассмотрение число $ex(p, H)$ "--- наибольшее число ребер, которое может быть в графе $G(V, E), |V| = p$, не содержащем подграф $H$. Следующая теорема решает задачу для случая $H = K_n$. 

\begin{theorem}[Турана]
	Имеет место следующее равенство
	$$ex(p, K_n) = \frac{(n-2)(p^2 - r^2)}{2(n-1)} + C_r^2.$$
\end{theorem}

	Это не единственный результат, которого добились исследователи теории графов. Были получены также такие равенства
	$$ex(p, C_p) = 1 + \frac{p(p+1)}{2}, \; ex(p, K_4 - x) = \lfloor \frac{p^2}{4}\rfloor, \; ex (p, K_{1, 3} + x) = \lfloor \frac{p^2}{4}\rfloor.$$

\mysubsection{Кликовое число}

	До этого мы фиксировали число вершин и меняли число ребер. Теперь зададимся другим вопросом: есть ли инварианты, которые, начиная с некоторого $n = |V|$, будут иметь какое-то заранее заданное значение?
	
	Например, можно рассмотреть \emph{кликовое число} для графа, которое соответствует размеру максимаьной клики (полного подграфа), а в кванторах выражается так:
	$$\omega(G) = \sup \lbrace |V'| \colon V' \subset V, \forall \!\ a, b \in V' \mapsto \lbrace a, b\rbrace \in V'\rbrace.$$

\mysubsection{Число независимости}

	\emph{Число независимости} двойственное число к кликовому числу, которое обозначает максимальное число вершин, которое можно выбрать из $V$ так, что никакие дву вершины не будут попарно соединены. В кванторах это число будет записываться так
	$$\alpha(G) = \sup \lbrace |V'| \colon V' \subset V, \forall \!\ a, b \in V' \mapsto \lbrace a, b\rbrace \notin V'\rbrace.$$

	Чтобы не лукавить, объясним, что мы подразумеваем под двойственностью в следующем утверждении.
	
\begin{statement}
	Пусть дан граф $G$, тогда имеет место два равенства
	$$\alpha(G) = \omega(\overline{G}), \; \omega(G) = \alpha(\overline{G}).$$
\end{statement}

	Искренне надеемся, что читатель сможет воспроизвести элементарное доказательство этого утверждения.
	
\mysubsection{Хроматическое число и хроматический индекс}
	
	Аналогично предыдущим инвариантам эти два имеют вполне понятную повседневную интерпретацию. Представьте, что перед вами граф, а вы маляр с кисточками и красками. Вам надо раскрасить либо все вершины так, чтобы никакие ребра не соединяли веришины одного цвета, либо все ребра так, чтобы ни в какой вершине не сходились ребра одного цвета. Первая раскраска называется \emph{правильной вершинной}, а вторая "--- \emph{правильной реберной}.
	
\begin{definition}
	Минимальное число красок нужное для раскраски графа $G$ правильной вершинной раскраской называется \emph{хроматическим числом} и обозначается $\chi(G)$. Минимальное число красок нужное для раскраски графа правильной реберной раскраской называется \emph{хроматическим индексом} и обозначается $\chi'(G)$. 
\end{definition}
	
	Эти два инварианта тоже можно сформулировать на языке кванторов:
	$$\chi(G) = \inf \lbrace \chi \colon V(G) = V_1 \sqcup V_2 \sqcup \dots \sqcup V_\chi \colon \forall \!\ i \in \overline{1, \chi}, a, b \in V_i \mapsto \lbrace a, b \rbrace \notin E(G)\rbrace,$$
	$$\chi'(G) = \inf \lbrace \chi' \colon E(G) = E_1 \sqcup E_2 \sqcup \dots \sqcup E_{\chi'} \colon \forall \!\ i \in \overline{1, \chi}, e, h \in E_i \mapsto \lbrace e \; not \; adj \; h\rbrace.$$
	
	В последней строке сокращением $adj$ мы заменили слово $adjacency$, которое буквально переводится как смежность ребер.
	
	Мы не зря только что упоминали о кликовых числах и числах независимости, потому что имеет место два простых ограничения на хроматическое число:
	$$\chi(G) \geqslant \frac{|V|}{\alpha(G)}, \; \chi(G) \geqslant \omega(G).$$
	Их достаточно просто получить из рассуждений. При дизъюнктном разбиении $V$ на независимые множества каждое из них имеет мощность не больше $\alpha(G)$, поэтому мощность их объединения, то есть просто число вершин, не больше $\alpha(G) \cdot \chi$. Вторая же оценка верна в силу того, что клику можно раскрасить только в число цветов, равное числу вершин клики.
	
\mysubsection{Задачи}

\begin{exersize}[Агаханов Н.Х., Богданов И.И., Кожевников П.А., Подлипский О.К., Терешин Д.А. <<Всероссийские олимпиады школьников по математике 1993~---~2009: Заключительные этапы>>]
	В стране $2001$ город, некоторые пары городов соединены дорогами, причем из каждого города выходит хотя бы одна дорога и нет города, соединенного дорогами со всеми остальными. Назовем множество городов $D$ \emph{доминирующим}, если любой не входящий в $D$ город соединен дорогой с одним из городов множества $D$. Известно, что в любом доминирующем множестве хотя бы $k$ городов. Докажите, что страну можно разбить на $2001-k$ республик так, что никакие два города из одной республики не будут соединены дорогой.
\end{exersize}	 

\begin{exersize}[Агаханов Н.Х., Богданов И.И., Кожевников П.А., Подлипский О.К., Терешин Д.А. <<Всероссийские олимпиады школьников по математике 1993~---~2009: Заключительные этапы>>]
	В стране $N$ городов. Некоторые пары из них соединены беспосадочными двусторонними авиалиниями. Оказалось, что для любого $k$ ($2 \leqslant k \leqslant N$) при любом выборе городов количество авиалиний между этими городами не будет превосходить $2k-2$. Докажите, что все авиалинии можно распределить между двумя авиакомпаниями так, что не будет замкнутого авиамаршрута, в котором все авиалинии принадлежат одной компании.
\end{exersize}	 