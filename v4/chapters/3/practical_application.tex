\section{О практическом применении}

	В этом параграфе мы поговорим о практическом применении теории графов. В том числе мы разберем возможные способы задания графа в ЭМВ и разберем нетривиальные примеры графов, на которых построены целые области математики.

\mysubsection{Способы задания графа}

	В первую очередь, разберемся с вопросом: как в памяти компьютера можно задать граф? Далее мы будем все появляющиеся структуры ассоциировать с графом на рисунке ниже.

\begin{center}\begin{tikzpicture}
	\Loop (-2, 2) [dir=WE];	
	
	\node at (0, 0)[below left]{$1$};  
	\node at (4, 0)[below right]{$2$};  
	\node at (4, 4)[above right]{$3$};  
	\node at (0, 4)[above left]{$4$};  
	\node at (-2, 1.8)[below]{$5$};
	
	\node at (2, 0.3){$a$};  
	\node at (2, -0.7){$b$};  
	\node at (3.3, 2){$c$};  
	\node at (4.7, 2){$d$};  
	\node at (2, 4.3){$e$};
	\node at (-1.2, 3.2){$f$};  
	\node at (-4.5, 2){$g$};  
	\node at (-0.8, 1.2){$h$};
	
	\node at (6, 0) {};  
	
	\tikzstyle{every node}=[circle, draw, fill=red, inner sep=0pt, minimum width=6pt]
	[line width=2pt]
	\draw (0, 0) edge [line width=1pt] (4, 0);
	\draw (0, 0) edge [bend right = 25, line width=1pt] (4, 0);
	\draw (4, 0) edge [bend right = 25, line width=1pt] (4, 4);
	\draw (4, 0) edge [bend left = 25, line width=1pt] (4, 4);
	\draw (4, 4) edge [line width=1pt] (0, 4);
	\draw (0, 4) edge [line width=1pt] (-2, 2);
	\draw (-2, 2) edge [line width=1pt] (0, 0);
	
	\node at (0, 0){};  
	\node at (4, 0){};  
	\node at (4, 4){};  
	\node at (0, 4){};  
	\node at (-2, 2){};
\end{tikzpicture}\end{center}
\begin{center}
	\small Рис. \images. <<Опытный образец>>
\end{center}	
	
	\textbf{Матрица инцидентности.} Вспомним, что граф определяется не только множествами $V, E$ вершин и ребер соответственно, но еще и отображением инцидентности. Это отображение можно представить в виде $I \colon E \to V \times V$. Однако для многих задач более удобно рассматривать отображение $I \colon V \to \BN_0$ в случае неориентированных графов и $I \colon V \to \BZ$ в случае орграфов.
	
	Тогда фактически образ каждого ребра при таком отображении будет вектор из целых чисел длины $|V|$. Удобной записью для них будет матрица, то есть таблица с числами, у которой каждый столбец соответствует образу какого-либо ребра.
		
	Ниже выписана матрица инцидентности для <<опытного образца>>.
		
\[	
	M_{inc} =
	\begin{pmatrix} 
	1 & 1 & 0 & 0 & 0 & 0 & 0 & 1 \\
	1 & 1 & 1 & 1 & 0 & 0 & 0 & 0 \\
	0 & 0 & 1 & 1 & 1 & 0 & 0 & 0 \\
	0 & 0 & 0 & 0 & 1 & 1 & 0 & 0 \\
	0 & 0 & 0 & 0 & 0 & 1 & 2 & 1 
	\end{pmatrix}
	\begin{array}{lr}
	1\\  
	2\\
	3\\
	4\\
	5
	\end{array}
\]
$$\; \; \; \; \; \, \, a \; \; \; b \; \; \; c \; \; \; d \; \; \; e \; \; \; f \; \; \; g \; \; \; h $$

	В данном случае, если ребро соединяет вершины $a$ и $b$, то в соответствующую строку были поставлены единицы. Как при подсчете степени вершины мы петлю учитывали дважды, так и в этой таблице мы пишем $2$ напротив той строки, которой соответствует вершина петли.
	
	Можно также вспомнить double counting (метод двойного подсчета), который во всей красе отражается в этой матрице: достаточно просуммировать сначала по всем строкам числа, а потом по всем столбцам.

	Остановим внимание читателя на том, что в случае простого графа в матрице инцилентности будут только единицы и нули. Случаи более сложных графов мы обсудим немного ниже.
	
	\textbf{Матрица смежности.} Опять дело пойдет о матрице, но теперь о более компактной. Предположим, что мы имеем дело с графом, в котором почти все ребра есть, тогда число столбцов в матрице инцидентности будет зависеть от количества вершин квадратично, а значит, нам придется хранить порядка $|V|^3$ чисел. Это не очень экономно... Прям очень неэкономно...
	
	Чтобы исправить это недоразумение, давайте будем задавать наш граф матрицей смежности $n \times n$, где $n$, как обычно, соответствует числу вершин в графе. Тогда на пересечение $i$ строки и $j$ столбца будет стоять число, равное количеству ребер, соединяющих $i$ вершину с $j$. Таким образом, мы даже покрываем множество мультиграфов заданием матрицы смежности.
	
	Для нашего <<опытного образца>> ниже представлена матрица смежности.

\[	
	M_{adj} =
	\begin{pmatrix} 
	0 & 2 & 0 & 0 & 1 \\
	2 & 0 & 2 & 0 & 0 \\
	0 & 2 & 0 & 1 & 0 \\
	0 & 0 & 1 & 0 & 1 \\
	1 & 0 & 0 & 1 & 1 
	\end{pmatrix}.
\]

	Примечательно, что след матрицы, то есть сумма его элементв на главной диагонали есть ни что иное, как число петель в графе.

	\textbf{Список смежности.} Заметим, что в предыдущем варианте мы независимо от числа ребер всегда тратили $|V|^2$ памяти только на хранение матрицы, что очень неэффективно, если мы имеем дело с почти пустым графом, в котором очень мало ребер. Чтобы исправить это недоразумение, можно хранить для каждой вершины список тех вершин, которые с ней смежны. Так, для <<опытного образца>> мы получим следующую конструкцию:
$$	1: 2, 5; $$
$$	2: 1, 3; $$
$$	3: 2, 4; $$
$$	4: 3, 5; $$
$$	5: 1, 4. $$

\mysubsection{Гиперграфы и случайные графы}

	Выше мы уже говорили про матрицы инцидентности. Оказывается, что можно на граф смотреть с точки зрения этой матрицы, а не наоборот, то есть такая матрица в целом несет полную информацию о графе. 
	
	В таком случае нам нужно уметь адаптировать матрицу инцидентности для других интересных классов графов. В случае с орграфами все очевидно: будем ставить на пересечения единицу, если ребро выходит из соответствующей вершины, и $-1$, если оно входит в соответствующую вершину.
	

\mysubsection{Автоматы как орграфы}

%\begin{center}\textbf{(Здесь будет инфа про матрицы смежности, списки смежности)}\end{center}
%\begin{center}\textbf{(Здесь будет инфа про алгоритмы на графах)}\end{center}
%\begin{center}\textbf{(Здесь будет инфа о экономике)}\end{center}
%\begin{center}\textbf{(Здесь будет инфа о конечных автоматах)}\end{center}