\section{Теория Рамсея}

\mysubsection{Задача Рамсея}

	Всегда разговор о теории Рамсея начинается со следующей задачи.
\begin{statement} [Задача Рамсея]
	Докажите, что среди шести человек найдутся либо три попарно знакомых, либо трое попарно незнакомых.
\end{statement}

	На теоретико-графовом языке эта же задача будет звучать так:
	
	\emph{либо в графе, либо в его дополнении будет треугольник, если число вершин не меньше $6$-ти,}
	
	или иначе:
	
	\emph{в графе на $n \geqslant 6$ вершинах число независимости или кликовое число не меньше трех.}

	Как мы уже говорили, экстремальные характеристики графа уже очень долго занимают умы сильнейших математиков. И одним из вопросов, стоящих в авангарде, является вопрос о минимальном графе по числу вершин, на котором некоторые характеристики достигают какого-то значения.
	
	Например, зададимся вопросом: начиная с какого $R$ в любом графе с $k \geqslant R$ вершинами будет либо клика размера $m$, либо независимое множество размера $n$? Это число $R$ получило название \emph{число Рамсея} и обозначается обычно $R(n, m)$. 

\begin{statement}
	Для чисел Рамсея верно рекуррентное соотношение
	$$R(n, m) \leqslant R(n, m-1) + R(n-1, m).$$
\end{statement}

\begin{consequence}
	Для чисел Рамсея имеет место следующая верхняя оценка
	$$R(n ,m) \leqslant C_{m+n-2}^{m-1}.$$
\end{consequence}

\mysubsection{Раскраска графов}

\mysubsection{Числа Рамсея большего порядка}

\mysubsection{Задачи}