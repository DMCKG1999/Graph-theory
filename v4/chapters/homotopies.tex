\section{Гомотопии}

\mysubsection{Деформационный ретракт}

	В предыдущем параграфе мы ввели понятие гомеоморфизма и обмолвились, что это отображние выполняет роль изоморфизма. В некоторых ситуациях нужно более сильное соответствие между рассматриваемыми пространствами, поэтому математики ввели понятие гомотопичности.
	
	Представим, что у нас есть воздушный шарик какой-то формы и мы его переводим непрерывной деформацией в другой шарик. Кажется, что мы опять говорим о гомеоморфизме, но на самом деле в следующее понятие вкладывается тот факт, что в конце мы не обязательно получим фигуру той же размерности, что и изначальная; она будет всего лишь предельной в этой последовательности деформаций.
	
\begin{definition}
	Будем называть семейство непрерывных от $t$ функций $\lbrace f_t \rbrace_{t \in [0, 1]}, f_t \colon X \to X$ \emph{гомотопией из $X$ в $Y \subset X$}, если выполнены три условия:
	\begin{itemize}
		\item $f_0 (X) = id_X$, то есть если взять <<начальное>> отображение, то оно будет действовать на множестве так же, как и тождественное;
		\item $f_t(Y) = id_Y \forall \!\ t \in [0, 1]$, то есть на самом множестве $Y$ все отображения стационарны;
		\item $f_1(X) = Y$, то есть <<последняя>> функция переводит все множество $X$ в $Y$.
	\end{itemize}
	При существовании гомотопии из $X$ в $Y$ последнее называют \emph{деформационным ретрактом}.
\end{definition}

	У одного и того же множества могут быть несколько деформационных ретрактов, которые могут быть не гомотопны, поэтому говорят, что \emph{$X$ гомотопически эквивалентно $Y$}, если существует пространство $Z$ такое, что и $X$, и $Y$ являются его деформационными ретрактами. Обозначается она так $X \simeq Y$.
	
	И тут начинается жесть.
	
\begin{definition}
	Два отображения $f$ и $g$ \emph{гомотопны}, если существует семейство непрерывных от $t$ функций $\lbrace f_t \rbrace_{t \in [0, 1]}, f_t \colon X \to X$, таких, что $f_0 = f$ и $f_1 = g$. Гомотопность отображений обозначается так же, как и эквивалентность $f \sim g$.
\end{definition}

\begin{testquestion}
	Докажите, что гомотопическая эквивалентность действительно задает отношение эквивалентности на множестве непрерывнх отображений.
\end{testquestion}

\begin{definition}
	Два пространства $X$ и $Y$ \emph{гомотопически эквивалентны}, если существуют два непрерывных отображения $f \colon X \to Y$ и $g \colon Y \to X$ такие, что $f \circ g \sim id_Y$ и $g \circ f \sim id_X$. 
\end{definition}
