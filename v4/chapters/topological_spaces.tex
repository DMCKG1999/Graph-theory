\section{Топологические пространства}

\mysubsection{Топология. Хаусдорфовость}

	Начнем мы с некоторого числа базовых понятий общей топологии, которые полезно знать даже просто для своего общего развития. Не стоит бояться их: они все достаточно простые.
	
\begin{definition}
	\emph{Топологическое пространство}~---~это пара $(X, \FO)$, в которой $X$~---~это произвольное множество, а $\FO \subset 2^{X}$, то есть является произвольным набором подмножеств множества $X$, и обладает следующими свойствами:
	\begin{itemize}
		\item $\varnothing \in \FO, X \in \FO$,
		\item $\forall \!\ A, B \in \FO \mapsto A \cup B \in \FO$,
		\item $\forall \!\ I, \lbrace A_i \rbrace_{i \in I} \colon \forall \;\ i \in I \mapsto A_i \in X \mapsto \bigcup\limits_{i \in I} A_i \in X$.
	\end{itemize}
	Все подмножества $X$, которые лежат в $\FO$, называются \emph{открытыми множествами}. Само множество $\FO$ называется \emph{топологией} и часто обозначается также буквой $\tau$.
\end{definition}

	Если объяснять простыми словами, то топологическое пространство~---~это множество, в котором мы выделили некоторые подмножества так, что для любых двух выделенных пересечение и (конечное, счетное или более, чем счетное) объединение будет тоже выделено.
	
\begin{example}
\begin{itemize}
	\item В $\BR^n$ можно задать стандартную топологию, в которой открытыми множествами будут открытые шары.
\end{itemize}
\end{example}

	Продолжая аналогии, мы в пространствах может ввести понятие пространств, в множествах~---~подмножества, а в топологических пространствах~---~\emph{топологическое подпространство}, то есть следующая пара множеств
	$$(Y, \lbrace Y \cap U \colon U \in \FO \rbrace), Y \in X.$$
	
	Например, в $\BR^n$ любое множество $Y$ в паре со всеми $Y \cap U$, то есть пересечениями с открытыми окрестностями (в том смысле, в котором рассказывают на курсах математического анализа) будет топологическим подпространством в $\BR^n$.

	Можно дать так же и следующие определения.
	
\begin{definition}
	Подмножество $A$ в топологическом пространстве $(X, \FO)$ называется \emph{закрытым}, если его дополнение открыто. \emph{Замыкание} подмножества $A$ назвается пересечение всех закрытых множеств, содержащих его. \emph{Границей множества $A$} называется пересечение замыкания множества и замыкания его дополнения. \emph{Внутренность} множества $A$~---~разность $A \ \partial A$.
\end{definition}

	У топологических пространств бывают очень интересные свойства, например, в некоторых пространствах есть такие точки $a, b$, что любое открытое множество, содержащее первую точку, будет содержать и вторую точку. Чтобы такого не было, ввели понятие \emph{хаусдорфовой топологии}~---~топологии, в которой для любых двух точек существуют открытые множества, которые между собой не пересекаются и содержат ровно по одной точке из них.
	
\begin{example}
	Чтобы читателю не казалось, что это свойство излишнее, предоставим пример нехаудорфово топологического пространства:
	$$(\BR, \varnothing \cup \BR \cup \lbrace [a, +\infty) \colon a \in \BR \rbrace).$$
	
	Проверяем, что это топологгическое пространство. А далее замечаем, что для любых двух чисел $a < b \mapsto b \in [a, +\infty)$, поэтому эта топология нехаусдорфова. 
\end{example}

\mysubsection{Непрерывные отображения}

	По понятным причинам мы не можем пользоваться непрерывным отображением, которое вводится на математическом анализе, потому что там есть метрика, а у нас ее еще нет. К тому же нам не подойдет обычное отображение, потому что для наших целей нужно, чтобы топологии <<нормально>> взаимодействовали с отображениями.
	
\begin{definition}
	Пусть $(X_1, \FO_1)$ и $(X_1, \FO_1)$~---~топологические пространства. Отображение $\varphi \colon X_1 \to X_2$ называется \emph{непрерывным}, если прообраз любого открытого множества открыто, то есть
	$$\forall \!\ Y \in \FO_2 \mapsto \varphi^{-1}(Y) \in \FO_1.$$
\end{definition}

	Далее мы будем подразумевать, что все пространства, с которыми мы работаем топологически, а все отображения непрерывны.

	Самое время вспомнить о критериях Вагнера и Куратовского, для записи которых нам пришлось ввести понятие гомоморфизма. На самом деле это понятие является наследником изоморфизма в топологическом мире.
	
\begin{definition}
	Отображение $\varphi \colon X_1 \to X_2$ из топологического пространства $(X_1, \FO_1)$ в топологическое пространство $(X_2, \FO_2)$ является \emph{гомоморфизмом}, если образ любого открытого множества из $\FO_1$ открыто и прообраз любого открытого множества из $\FO_2$ тоже открыто.
\end{definition}

	Соответственно гомеоморфные пространства мы будем обозначать так $X_1 \cong X_2$.
		
\mysubsection{Задачи}
\begin{exersize}[J. Matoušek. <<Using the Borsuk-Ulam theorem: Lectures on topological methods in combinatorics and geometry>>]
	Проверьте следующие гомеоморфизмы:
	\begin{enumerate}
		\item $\BR \cong (0, 1) \cong (S^1 \ \lbrace(0,1)\rbrace)$;
		\item $S^1 \cong \partial([0,1]^2)$.
	\end{enumerate}
\end{exersize}
